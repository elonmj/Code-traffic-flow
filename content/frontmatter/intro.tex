\chapter*{Introduction générale}
\addcontentsline{toc}{chapter}{Introduction générale}
\label{chap:intro}

% Section 1: Contexte et justification de l'étude
\section{Contexte et justification de l'étude}
\label{sec:contexte_justification}

La mobilité en milieu urbain constitue un réel défi pour le développement durable des villes d'Afrique de l'Ouest. La croissance démographique rapide, l'urbanisation accrue ont entraîné une augmentation significative du trafic routier dans les grandes agglomérations de la région. Cotonou (Bénin), Abidjan (Côte d'Ivoire), Dakar (Sénégal) et Lagos (Nigeria) sont des exemples emblématiques de cette problématique, où les embouteillages quotidiens affectent non seulement la productivité économique mais aussi la qualité de vie des citoyens.

Dans ce contexte, le Bénin, et plus particulièrement la ville de Cotonou, présente des caractéristiques spécifiques qui rendent la gestion du trafic particulièrement complexe. L'hétérogénéité extrême du parc roulant, marquée par la prédominance écrasante des deux-roues motorisés, constitue un défi unique pour les modèles de trafic classiques. Ces véhicules, qui assurent une part majeure des déplacements urbains, adoptent des comportements spécifiques tels que le "gap-filling", l'"interweaving" et le "creeping", influençant profondément la dynamique globale du flux. De plus, l'infrastructure routière, composée de routes bitumées, pavées et de nombreuses voies en terre, souvent dans un état de dégradation variable, ajoute une couche de complexité supplémentaire.

Face à ces défis, les solutions traditionnelles de gestion du trafic, basées sur des modèles statiques et des politiques standardisées ne sont pas adaptées à ce contexte. Il devient donc impératif de développer des outils plus adaptés aux réalités locales, capables de modéliser finement la dynamique du trafic et d'optimiser les systèmes de contrôle en temps réel.

% Section 2: But de la Recherche
\section{But de la Recherche}
\label{sec:but_recherche}

Le but principal de cette recherche est de développer et d'appliquer un jumeau numérique de trafic macroscopique basé sur un modèle ARZ (Aw-Rascle-Zhang) étendu, spécifiquement adapté aux contextes urbains d'Afrique de l'Ouest, et un système d'optimisation par intelligence artificielle (apprentissage par renforcement) intégrant les spécificités comportementales régionales. Cette approche vise à fournir une solution innovante pour la gestion intelligente du trafic dans des environnements caractérisés par une forte hétérogénéité du trafic et des données limitées.

% Section 3: Problème et Questions de Recherche
\section{Problème et Questions de Recherche}
\label{sec:probleme_questions}

Le problème central de cette recherche est de répondre à la question suivante : Comment modéliser efficacement le trafic urbain en Afrique de l'Ouest en intégrant les caractéristiques du trafic ouest-africain dans un environnement de données disponibles (corridor Lagos), et comment optimiser ce trafic avec des outils d'intelligence artificielle adaptés ?

Pour adresser ce problème, nous formulons les questions de recherche suivantes :
\begin{enumerate}
    \item Comment étendre le modèle ARZ pour capturer les comportements spécifiques des motocyclettes et l'hétérogénéité du trafic en Afrique de l'Ouest ?
    \item Comment calibrer et valider un jumeau numérique de trafic dans un contexte de données limitées ?
    \item Comment concevoir un agent d'apprentissage par renforcement capable d'optimiser les feux de signalisation dans ce contexte spécifique ?
\end{enumerate}

% Section 4: Objectifs du Mémoire
\section{Objectifs du Mémoire}
\label{sec:objectifs_memoire}

\subsection{Objectif général}
\label{subsec:objectif_general}

Créer et valider un jumeau numérique ARZ adapté aux contextes ouest-africains et un agent d'apprentissage par renforcement (RL) pour l'optimisation des feux de signalisation.

\subsection{Objectifs spécifiques}
\label{subsec:objectifs_specifiques}

\begin{enumerate}
    \item Développer un modèle ARZ étendu capturant les comportements spécifiques des motocyclettes observés en Afrique de l'Ouest.
    \item Développer une chaîne numérique haute-fidélité pour le corridor Victoria Island (Lagos) en utilisant des méthodes numériques avancées (WENO).
    \item Calibrer et valider le jumeau numérique sur le corridor de Victoria Island (Lagos) avec les données disponibles.
    \item Concevoir et entraîner un agent d'apprentissage par renforcement dans l'environnement simulé pour optimiser les feux de signalisation.
    \item Évaluer les performances du système d'optimisation dans divers scénarios de trafic.
\end{enumerate}

% Section 5: Résultats Attendus et Cibles
\section{Résultats Attendus et Cibles}
\label{sec:resultats_attendus}

Cette recherche vise à obtenir les résultats suivants :
\begin{itemize}
    \item Un jumeau numérique de trafic validé, basé sur un modèle ARZ étendu, capable de simuler fidèlement la dynamique du trafic en Afrique de l'Ouest.
    \item Un agent d'apprentissage par renforcement performant pour l'optimisation des feux de signalisation, démontrant des gains significatifs par rapport aux méthodes classiques.
    \item Un modèle validé sur Lagos, démontrant son potentiel d'adaptation à d'autres contextes ouest-africains.
\end{itemize}

% Section 6: Limites de l'Étude
\section{Limites de l'Étude}
\label{sec:limites_etude}

Cette étude présente certaines limites :
\begin{itemize}
    \item \textbf{Contraintes liées aux données :} La disponibilité et la qualité des données de trafic peuvent varier, notamment pour la calibration et la validation du modèle.
    \item \textbf{Validation sur un seul corridor :} Les résultats sont validés sur le corridor Victoria Island (Lagos), ce qui peut limiter la généralisation à d'autres contextes.

\end{itemize}

% Section 7: Nouveauté de l'Étude
\section{Nouveauté de l'Étude}
\label{sec:nouveaute_etude}

L'originalité de cette recherche réside dans la combinaison de trois piliers innovants :
\begin{enumerate}
    \item \textbf{Modèle ARZ adapté :} Première application d'un modèle ARZ étendu spécifiquement conçu pour capturer les spécificités du trafic ouest-africain, notamment l'hétérogénéité extrême et les comportements spécifiques des motocyclettes.
    \item \textbf{Modèle multi-classes adapté aux spécificités régionales ouest-africaines} : Développement d'une approche méthodologique originale permettant de modéliser les interactions complexes dans le trafic ouest-africain.
    \item \textbf{Optimisation par RL :} Couplage innovant du modèle ARZ étendu avec un agent d'apprentissage par renforcement pour l'optimisation intelligente du trafic dans un contexte urbain complexe.
\end{enumerate}

% Section 8: Grandes Divisions du Mémoire
\section{Grandes Divisions du Mémoire}
\label{sec:divisions_memoire}

Ce mémoire est structuré en trois parties principales :

\begin{itemize}
    \item \textbf{Partie I : Revue de Littérature et Fondements Scientifiques} \\
          Cette partie présente l'état de l'art des modèles de trafic macroscopiques (LWR, ARZ), les méthodes numériques associées, et les principes de l'apprentissage par renforcement appliqué à la gestion du trafic. Elle met en évidence les spécificités du trafic urbain en Afrique de l'Ouest et justifie l'approche méthodologique choisie.

    \item \textbf{Partie II : Matériels et Méthodes – Développement du Jumeau Numérique et de l'Environnement RL} \\
          Cette partie détaille la formulation mathématique du modèle ARZ étendu, la conception et l'implémentation de la chaîne numérique haute-fidélité, ainsi que la calibration du jumeau numérique et la préparation de l'environnement d'apprentissage par renforcement.

    \item \textbf{Partie III : Résultats, Validation et Discussion} \\
          Cette partie présente la validation du jumeau numérique, l'entraînement de l'agent intelligent, l'évaluation des performances et l'analyse de la robustesse du système d'optimisation. Elle se termine par une discussion générale et des perspectives de recherche.
\end{itemize}

% \begin{figure}[htbp]
%     \centering
%     \includegraphics[width=0.8\textwidth]{images/intro/schema_structure_memoire.png}
%     \caption{Structure générale du mémoire}\label{fig:structure_memoire}
% \end{figure}