% Résumé de la thèse en français
% Environ 500 mots ou moins

\noindent
La congestion urbaine représente l'un des défis majeurs des métropoles africaines en pleine croissance. Lagos, avec ses millions de véhicules et son trafic hautement hétérogène (motos, voitures, bus, camions), illustre de manière exemplaire cette problématique. Les systèmes de gestion du trafic traditionnels, basés sur des feux de signalisation à temps fixe, s'avèrent inadaptés face à la complexité et à la variabilité de ces flux.

Cette thèse propose une approche novatrice combinant modélisation mathématique avancée et intelligence artificielle pour optimiser la gestion du trafic dans des contextes de forte hétérogénéité. Nous développons un cadre théorique rigoureux basé sur les lois de conservation hyperboliques multi-classes, permettant de capturer la dynamique des différentes catégories de véhicules et leurs interactions. Le modèle proposé généralise les approches classiques (LWR, ARZ) en intégrant explicitement l'hétérogénéité du trafic africain.

La méthodologie s'articule autour de trois contributions majeures. Premièrement, nous formulons un modèle mathématique multi-classes avec couplage aux intersections, en développant des solveurs de Riemann adaptés et des schémas numériques de type Godunov garantissant la conservation de la masse et la positivité des densités. Deuxièmement, nous construisons un jumeau numérique du corridor de Victoria Island à Lagos, calibré et validé sur des données réelles, offrant une plateforme de simulation haute-fidélité. Troisièmement, nous concevons un environnement d'apprentissage par renforcement conforme à l'API Gymnasium, permettant l'entraînement d'agents intelligents pour le contrôle adaptatif des feux de signalisation.

Les résultats démontrent la capacité du modèle à reproduire fidèlement les comportements observés (diagrammes fondamentaux, ondes de choc, phénomènes de raréfaction) et l'efficacité de l'approche par apprentissage par renforcement pour réduire significativement les temps d'attente et améliorer la fluidité du trafic. Cette recherche ouvre des perspectives prometteuses pour le déploiement de systèmes de gestion intelligente du trafic dans les villes africaines.
