\section{Modèle Réseau : Intersections et Couplage des Flux}
\label{sec:modeles_intersections}

\subsection{Introduction : Le Défi des Intersections pour les Modèles de Second Ordre}
L'application du modèle ARZ multi-classes à un réseau urbain réaliste nécessite une modélisation rigoureuse des intersections. Celles-ci constituent des points de rupture où la dynamique unidimensionnelle de chaque segment est mise à l'épreuve. Pour un modèle de second ordre comme ARZ, le défi est double : il faut non seulement conserver la masse des véhicules, mais aussi définir comment la variable lagrangienne $w$, qui représente l'état comportemental du conducteur, est transmise à travers la complexité d'un nœud \cite{GaravelloPiccoli2006}.

Cette section présente un cadre de modélisation unifié capable de gérer les types d'intersections les plus courants (carrefours giratoires, jonctions non signalisées, carrefours à feux) dans le contexte ouest-africain. Notre approche se distingue par :
\begin{itemize}
    \item Un \textbf{formalisme unifié} en deux étapes (flux et comportement) pour tous les types de nœuds.
    \item La prise en compte de \textbf{comportements spécifiques aux motos}, comme le \textit{creeping} en congestion.
\end{itemize}
L'objectif est de construire un modèle de réseau physiquement cohérent, numériquement stable, et suffisamment riche pour servir de base à l'optimisation des feux de signalisation par apprentissage par renforcement.

\subsection{Cadre Unifié de Modélisation des Nœuds}
Pour assurer la cohérence du modèle de réseau, nous adoptons un cadre conceptuel unifié en deux étapes pour tous les types d'intersections, inspiré de \cite{HoldenRisebro2015, AndreianovPanov2012}.

\subsubsection{Étape 1 : Détermination des Flux de Masse via la Logique Demande-Offre}
Le flux de véhicules $ q_{ij}^k $ de l'arc entrant $ i $ vers l'arc sortant $ j $ pour la classe $ k $ est déterminé par la compétition entre la demande de l'amont et l'offre de l'aval \cite{Daganzo1995, Lebacque1996}. Le flux effectif est le minimum entre la demande des véhicules souhaitant traverser et la capacité de la route en aval à les accepter :
\[
    q_{ij}^k = \min\left(D_i^k(\rho_i^k, v_i^k), S_j^k(\rho_j^k)\right) \cdot \beta_{ij}^k
\]
où $ D_i^k $ est la demande, $ S_j^k $ est l'offre, et $ \beta_{ij}^k $ est le coefficient de répartition du trafic \cite{CocliteGaravelloPiccoli2005}. Ce qui change d'un type d'intersection à l'autre, ce sont les \textbf{règles de priorité} (par exemple, un feu rouge impose $D_i^k = 0$) qui modulent la "demande autorisée" à un instant donné \cite{BressonPiccoli2019}.

\subsubsection{Étape 2 : Transmission du Comportement via un Couplage Phénoménologique}
Une fois les flux de masse établis, la transmission de la variable lagrangienne $ w $ est décrite par une condition de couplage phénoménologique. Cette approche évite la complexité des solveurs de Riemann théoriques \cite{GaravelloPiccoli2006} tout en offrant une flexibilité essentielle. La variable $ w $ à l'entrée d'un arc sortant est une combinaison de l'état d'équilibre local et de l'état entrant :
\[
    w_{\text{out}}^k = \left(V_{e,k}(\rho_{\text{out}}^k) + p_k(\rho_{\text{out}}^k)\right) + \theta_k \cdot \left(w_{\text{in}}^k - \left(V_{e,k}(\rho_{\text{in}}^k) + p_k(\rho_{\text{in}}^k)\right)\right)
\]
Le paramètre de couplage $ \theta_k \in [0,1] $ représente le degré de "mémoire comportementale" conservée par la classe $ k $ en franchissant l'intersection. Une valeur de $\theta_k=0$ signifie une réinitialisation complète du comportement (par exemple, après un long arrêt), tandis qu'une valeur de $\theta_k=1$ implique une conservation parfaite du comportement (conduite fluide) \cite{HertyKlar2003}. La stabilité numérique de ce couplage est assurée par un critère CFL adapté (voir Annexe \ref{annexe:demonstration_riemann}).

\subsection{Spécialisation du Modèle et Comportements Émergents}
Le cadre unifié est spécialisé en ajustant les règles de priorité et les paramètres de couplage $\theta_k$ pour chaque scénario.

\subsubsection{Spécialisation par Type d'Intersection}
\begin{itemize}
    \item \textbf{Carrefour Giratoire :} La priorité est à l'anneau. Le paramètre $\theta_k$ est faible pour l'insertion (adaptation forte, $\theta_k \approx 0.2$), élevé pour la circulation et la sortie (mouvement fluide, $\theta_k \approx 0.8$).
    \item \textbf{Jonction Non Signalisée (Stop/Cédez-le-passage) :} La hiérarchie est stricte. Le flux sur l'axe principal est peu perturbé ($\theta_{\text{principal}} \approx 0.9$), tandis que le flux secondaire subit une réinitialisation quasi complète de son état dynamique ($\theta_{\text{secondaire}} \approx 0.1$).
    \item \textbf{Carrefour à Feux :} C'est le cas le plus structuré. En phase rouge, la demande autorisée est nulle ($D_i^k = 0$). En phase verte, le paramètre $\theta_k$ modélise \textbf{l'agressivité de l'accélération au démarrage}. Les valeurs sont plus élevées pour les motos (démarrage vif, $\theta_{\text{moto}} \in [0.7, 0.9]$) que pour les voitures (démarrage modéré, $\theta_{\text{voiture}} \in [0.4, 0.6]$).
\end{itemize}

\subsubsection{Gestion des Intersections Saturées et Phénomène de \textit{Creeping}}
Dans le contexte ouest-africain, les intersections sont souvent saturées. Notre modèle intègre la capacité des motos à continuer de progresser (\textit{creeping}) même lorsque la densité des voitures atteint son maximum ($\rho_{\text{jam},c}$). Lorsque $\rho_c \geq \rho_{\text{jam},c}$, les motos peuvent encore traverser l'intersection avec un débit résiduel $q_{\text{creep}}^{\text{moto}}$, modélisé comme une fonction de la densité des motos et inhibé par la présence des voitures arrêtées \cite{FanWork2015}.

% ============================================================================
% NOTE: Section déplacée vers perspectives (section9_discussion_perspectives.tex)
% Cette stratégie de calibration hybride via SUMO n'a pas pu être réalisée
% dans le cadre de ce travail. Elle représente une approche méthodologique
% importante pour les travaux futurs de calibration du modèle.
% ============================================================================

% \subsection{Stratégie de Calibration Hybride via Simulation}
% La calibration des paramètres du modèle, en particulier $\theta_k$, est un défi en l'absence de données de trajectoires détaillées pour le contexte nigérian. Nous adoptons donc une stratégie de calibration hybride en utilisant le simulateur de trafic microscopique \textbf{SUMO} comme un \textbf{laboratoire numérique} \cite{KrajzewiczEtAl2012}. L'objectif n'est pas de répliquer une intersection réelle, mais de créer un \textbf{banc d'essai numérique avec une dynamique plausible} pour étudier le lien entre les comportements microscopiques et les paramètres de notre modèle macroscopique.

% La méthodologie est la suivante :
% \begin{enumerate}
%     \item \textbf{Configuration de SUMO :} Le simulateur est configuré pour refléter les caractéristiques qualitatives du trafic (accélération, temps de réaction, etc.) pour les motos et les voitures.
%     \item \textbf{Génération de Données de Référence :} Nous simulons plusieurs cycles de feux et des détecteurs virtuels extraient les données macroscopiques (densité, vitesse) pour chaque classe à l'entrée et à la sortie du carrefour.
%     \item \textbf{Calibration de $\theta_k$ :} La variable $w$ de notre modèle, absente dans SUMO, est reconstruite à partir des données simulées. Le paramètre $\theta_k$ est alors optimisé pour minimiser l'écart entre le $w$ prédit par notre modèle de couplage et le $w$ reconstruit à partir de SUMO.
% \end{enumerate}
% Cette approche, inspirée des recherches sur les modèles hybrides \cite{HertyKolbe2022}, nous permet d'obtenir des valeurs de paramètres physiquement pertinentes malgré la rareté des données.

\subsection{Synthèse de la Section}
Cette section a établi un cadre de modélisation mathématique à la fois unifié et flexible pour la gestion des intersections dans un modèle de trafic ARZ multi-classes. L'approche en deux étapes (demande-offre pour les flux, couplage phénoménologique pour le comportement) permet de traiter de manière cohérente divers types d'intersections tout en capturant les spécificités du trafic ouest-africain, notamment le comportement agile des motos et le phénomène de \textit{creeping}.

Cette modélisation des intersections, physiquement cohérente et numériquement stable, constitue la pierre angulaire de notre jumeau numérique et la base sur laquelle l'agent d'apprentissage par renforcement pourra optimiser efficacement les stratégies de feux de signalisation dans les chapitres suivants. La calibration des paramètres comportementaux, notamment $\theta_k$, reste un défi important qui devra être abordé dans les travaux futurs avec des données de terrain plus détaillées.

\begin{keypointsbox}[Points Clés du Chapitre 4 (Révisé)]
    \begin{itemize}
        \item \textbf{Modélisation unifiée des nœuds :} Un cadre unique (Demande-Offre + Couplage $\theta_k$) pour tous les types d'intersections.
        \item \textbf{Couplage phénoménologique :} Le paramètre $\theta_k$ capture la "mémoire comportementale" des conducteurs, spécialisé pour chaque situation (démarrage au feu, insertion, etc.).
        \item \textbf{Spécificités motos intégrées :} Le modèle gère explicitement l'agilité et le \textit{creeping} des motos en conditions de saturation.
        \item \textbf{Fondation pour l'optimisation :} Le modèle de réseau résultant est prêt à être utilisé pour l'entraînement de l'agent RL.
    \end{itemize}
\end{keypointsbox}
