\section{Discussion Générale et Perspectives}
\label{sec:discussion_perspectives}

\subsection{Introduction}
% Contenu de l'introduction

\subsection{Synthèse des Principaux Résultats}
\subsubsection{Contributions Méthodologiques}
% Contenu de la sous-section

\subsubsection{Contributions Techniques}
% Contenu de la sous-section

\subsubsection{Validation Expérimentale}
% Contenu de la sous-section

\subsection{Discussion Critique des Résultats}
\subsubsection{Forces de l'Approche Globale}
% Contenu de la sous-section

\subsubsection{Faiblesses et Limitations Identifiées}
% Contenu de la sous-section

\subsection{Généralisabilité Régionale et Perspectives de Déploiement}
\subsubsection{Applicabilité aux Autres Villes Ouest-Africaines}
% Contenu de la sous-section

\subsubsection{Stratégie de Déploiement : De Lagos vers Cotonou}
% Contenu de la sous-section

% NOTE ÉDITORIALE : Cette section doit faire référence à l'analyse d'impact sociétal et économique
% développée dans la section \ref{subsec:impact_socioeconomique} de la section 7, particulièrement
% pour justifier les priorités de déploiement et les modèles de financement basés sur le ROI démontré.

\subsubsection{Méthodologie de Transposition comme Perspective Future}
Le concept de transposition régionale, initialement envisagé comme une composante de notre méthodologie, a été réorienté comme un axe de recherche futur majeur. L'idée est de développer une méthodologie formelle pour adapter les modèles et les paramètres calibrés sur un site riche en données (comme Lagos) à un site avec des données plus rares (comme Cotonou).

Cette approche reposerait sur les étapes suivantes :
\begin{enumerate}
    \item \textbf{Identification des invariants régionaux} : Déterminer les paramètres comportementaux (par exemple, les fonctions de pression, les temps de réaction relatifs) qui sont stables à travers la région ouest-africaine.
    \item \textbf{Développement de fonctions de transfert} : Créer des modèles statistiques ou basés sur l'apprentissage pour mapper les caractéristiques d'infrastructure (qualité de route, densité d'intersections) aux paramètres du modèle de trafic.
    \item \textbf{Calibration bayésienne} : Utiliser les paramètres de Lagos comme a priori dans un processus de calibration bayésien pour le modèle de Cotonou, permettant une convergence rapide même avec peu de données locales.
    \item \textbf{Apprentissage par transfert (Transfer Learning)} : Entraîner un agent RL sur le jumeau numérique de Lagos, puis affiner (fine-tune) cet agent sur un jumeau numérique simplifié de Cotonou.
\end{enumerate}

Cette perspective de recherche est prometteuse pour accélérer le déploiement de solutions de gestion intelligente du trafic dans de nombreuses villes africaines où la collecte de données à grande échelle reste un défi.

\subsubsection{Besoins en Données Locales pour l'Adaptation}
% Contenu de la sous-section

\subsection{Axes de Recherche Futurs}
\subsubsection{Extensions du Modèle}
% Contenu de la sous-section

\subsubsection{Amélioration de la Méthodologie de Calibration}
\label{subsec:amelioration_calibration}

L'approche de calibration déterministe utilisée dans cette thèse, bien que fonctionnelle, présente des limitations importantes identifiées à la Section~\ref{sec:validation_entrainement}. L'état de l'art récent en modélisation du trafic (2022-2025) suggère plusieurs axes d'amélioration majeurs qui constitueraient une extension naturelle de ce travail.

\paragraph{Calibration Bayésienne avec Quantification d'Incertitude}
La recherche récente en modélisation macroscopique du trafic s'oriente vers des approches probabilistes qui traitent explicitement les incertitudes sur les paramètres et les mesures. Cette évolution méthodologique présenterait plusieurs avantages critiques pour notre modèle ARZ étendu :

\paragraph{Cadre bayésien pour les paramètres comportementaux}
\begin{itemize}
    \item \textbf{Distributions de paramètres plutôt que valeurs ponctuelles} : Utiliser des frameworks comme PyMC3 ou NumPyro pour estimer des distributions de probabilité complètes pour chaque paramètre ($\alpha$, $\tau_m$, $\rho_{jam,m}$, etc.), permettant de quantifier la confiance dans les estimations.
    \item \textbf{Incorporation de la connaissance a priori} : Intégrer les connaissances physiques (e.g., $0 < \alpha < 1$, $\tau_m > 0$) et les observations qualitatives (e.g., comportements des zémidjans) comme distributions a priori informatives.
    \item \textbf{Propagation d'incertitude} : Calculer rigoureusement comment les incertitudes sur les paramètres se propagent aux prédictions du modèle, fournissant des intervalles de confiance sur les vitesses et densités prédites.
    \item \textbf{Calibration hiérarchique} : Modéliser explicitement les variations des paramètres selon le segment routier, le moment de la journée, ou les conditions météorologiques via des modèles bayésiens hiérarchiques.
\end{itemize}

\paragraph{Bénéfices pour le contexte ouest-africain}
Cette approche serait particulièrement pertinente pour notre cas d'étude car :
\begin{itemize}
    \item Les données TomTom présentent des incertitudes variables selon les zones (meilleure couverture dans les corridors principaux).
    \item Les comportements des motos montrent une variabilité inhérente difficile à capturer par des paramètres déterministes.
    \item Les décisions de déploiement bénéficieraient de connaître la robustesse des prédictions face aux incertitudes paramétriques.
\end{itemize}

\paragraph{Contraintes Physiques Informées par le Modèle (PINNs)}
Une seconde limitation majeure de l'approche actuelle est l'absence de contraintes explicites garantissant la cohérence physique des paramètres calibrés. Les Physics-Informed Neural Networks (PINNs) et les approches d'optimisation contrainte offrent des solutions prometteuses :

\paragraph{Intégration de contraintes théoriques}
\begin{itemize}
    \item \textbf{Conservation de la masse rigoureuse} : Imposer que les paramètres calibrés garantissent la conservation stricte du nombre de véhicules à toutes les échelles spatiales et temporelles.
    \item \textbf{Respect de l'anisotropie ARZ} : Contraindre les paramètres pour assurer que $\lambda_1 = v - \rho p'(\rho) \leq v = \lambda_2$ (les perturbations se propagent vers l'arrière).
    \item \textbf{Positivité des vitesses} : Garantir formellement que $V_{e,i}(\rho) \geq V_{creeping} \geq 0$ pour toute densité physiquement admissible.
    \item \textbf{Monotonicité de la fonction de pression} : Imposer $p'(\rho) > 0$ pour éviter les comportements non physiques identifiés dans la littérature ARZ.
\end{itemize}

\paragraph{Méthodes d'implémentation}
\begin{itemize}
    \item Optimisation contrainte via multiplicateurs de Lagrange augmentés pour les contraintes non linéaires.
    \item Pénalisation différentiable des violations de contraintes physiques dans la fonction objectif.
    \item Utilisation de PINNs pour apprendre simultanément les paramètres et vérifier la cohérence avec les EDP du modèle ARZ.
\end{itemize}

\paragraph{Optimisation Multi-Objectifs}
La calibration actuelle minimise exclusivement l'erreur sur les vitesses (RMSE). Une approche multi-objectifs plus riche permettrait de :

\begin{itemize}
    \item \textbf{Équilibrer vitesse et congestion} : Optimiser simultanément la précision sur les vitesses moyennes, les queues maximales, et les temps de traversée.
    \item \textbf{Capturer différents régimes de trafic} : Assurer de bonnes performances à la fois en flux libre, en congestion modérée, et en saturation extrême.
    \item \textbf{Compromis précision/complexité} : Trouver le juste équilibre entre fidélité du modèle et parcimonie paramétrique via des approches Pareto-optimales.
    \item \textbf{Intégrer objectifs opérationnels} : Calibrer directement pour des métriques pertinentes pour le contrôle RL (réduction des temps d'attente, équité entre classes de véhicules).
\end{itemize}

\paragraph{Validation Probabiliste Avancée}
Au-delà des simples partitions train/test, des méthodes de validation plus rigoureuses devraient être adoptées :

\begin{itemize}
    \item \textbf{Validation croisée spatio-temporelle} : Respecter les corrélations temporelles des données de trafic via des protocoles de validation chronologique.
    \item \textbf{Bootstrapping et rééchantillonnage} : Estimer la stabilité des paramètres calibrés face à la variabilité des données d'entraînement.
    \item \textbf{Tests de généralisation out-of-distribution} : Valider explicitement sur des conditions non vues pendant la calibration (événements exceptionnels, périodes atypiques).
    \item \textbf{Analyse de sensibilité globale} : Utiliser des méthodes comme Sobol pour identifier systématiquement les paramètres les plus influents et prioriser les efforts de collecte de données.
\end{itemize}

\paragraph{Feuille de Route pour l'Implémentation}
Une modernisation progressive de l'approche de calibration pourrait suivre les phases suivantes :

\begin{enumerate}
    \item \textbf{Phase 1 (Court terme, 6-12 mois)} : Migration vers une calibration bayésienne simple (PyMC3) pour les 3-4 paramètres les plus critiques ($\alpha$, $\tau_m$, $\rho_{jam,m}$), avec visualisation des distributions postérieures.
    \item \textbf{Phase 2 (Moyen terme, 12-18 mois)} : Intégration de contraintes physiques explicites (conservation, anisotropie) via optimisation contrainte, validation sur cas tests théoriques.
    \item \textbf{Phase 3 (Long terme, 18-24 mois)} : Développement d'une approche multi-objectifs complète avec fronts de Pareto, déploiement de PINNs pour calibration auto-cohérente.
    \item \textbf{Phase 4 (Perspective, 24+ mois)} : Système de calibration adaptatif en temps réel utilisant les données collectées en continu pour mettre à jour les distributions de paramètres.
\end{enumerate}

Cette évolution méthodologique transformerait la calibration d'une étape préliminaire déterministe en un processus scientifiquement rigoureux et opérationnellement robuste, essentiel pour un déploiement fiable dans les environnements urbains ouest-africains.

\subsubsection{Améliorations de l'IA}
% Contenu de la sous-section

\subsubsection{Validation Étendue}
% Contenu de la sous-section

\subsection{Limites et Remarques Finales}
Cette thèse a développé une approche intégrée innovante pour l'optimisation du trafic urbain en Afrique de l'Ouest, combinant modélisation ARZ étendue et apprentissage par renforcement. Les résultats obtenus ouvrent des perspectives prometteuses tout en identifiant clairement les axes d'amélioration pour les travaux futurs.

\begin{keypointsbox}[Points Clés du Chapitre 9]
    \begin{itemize}
        \item \textbf{Synthèse réussie} : Contributions méthodologiques, techniques et expérimentales validées
        \item \textbf{Analyse critique} : Forces/faiblesses identifiées pour amélioration continue
        \item \textbf{Généralisabilité confirmée} : Applicabilité démontrée autres contextes ouest-africains
        \item \textbf{Stratégie déploiement} : Feuille route Lagos → Cotonou → Extension régionale
        \item \textbf{Perspectives riches} : Axes recherche futurs extensions modèle, IA, validation
        \item \textbf{Impact global} : Vers jumeau numérique west-africain scalable et socialement bénéfique
    \end{itemize}
\end{keypointsbox}