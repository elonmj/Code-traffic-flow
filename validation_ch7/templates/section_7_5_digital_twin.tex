% Template pour Section 7.5 - Validation Jumeau Numérique
% Auto-generated by validation framework

\subsection{Validation du Jumeau Numérique}

Cette section démontre la capacité du jumeau numérique à reproduire fidèlement les comportements observés et à fournir des prédictions fiables.

\subsubsection{Conservation des Propriétés Physiques}

\paragraph{Conservation de Masse Multi-Classes}

La validation rigoureuse de la conservation de masse pour chaque classe de véhicules :

\begin{table}[H]
    \centering
    \caption{Validation conservation de masse par classe}
    \label{tab:mass_conservation}
    \begin{tabular}{|l|c|c|c|c|}
        \hline
        \textbf{Classe véhicule} & \textbf{Erreur relative} & \textbf{Seuil} & \textbf{Statut}     & \textbf{Durée test}     \\
        \hline
        Motocycles               & {mass_error_moto:.2e}    & < 1e-6         & {mass_status_moto}  & {test_duration_moto} s  \\
        Voitures                 & {mass_error_car:.2e}     & < 1e-6         & {mass_status_car}   & {test_duration_car} s   \\
        Total combiné            & {mass_error_total:.2e}   & < 1e-6         & {mass_status_total} & {test_duration_total} s \\
        \hline
    \end{tabular}
\end{table}

\paragraph{Validation des Flux aux Interfaces}

Conservation des flux entre segments du réseau :

\begin{itemize}
    \item Interface entrée-sortie : Écart relatif < {flux_interface_error:.2e}
    \item Conditions limites absorbantes : Réflexion < {boundary_reflection:.2e}
    \item Cohérence temporelle : Dérive < {temporal_drift:.2e}/h
\end{itemize}

\subsubsection{Reproduction des Comportements Observés}

\paragraph{Patterns de Trafic Caractéristiques}

Validation de la reproduction des comportements types du trafic urbain :

\begin{table}[H]
    \centering
    \caption{Reproduction patterns de trafic observés}
    \label{tab:traffic_patterns}
    \begin{tabular}{|l|c|c|c|}
        \hline
        \textbf{Pattern observé}    & \textbf{Corrélation R²} & \textbf{Seuil} & \textbf{Statut}      \\
        \hline
        Formation congestion        & {congestion_r2:.3f}     & > 0.8          & {congestion_status}  \\
        Dissolution embouteillage   & {dissolution_r2:.3f}    & > 0.8          & {dissolution_status} \\
        Écoulement libre            & {freeflow_r2:.3f}       & > 0.8          & {freeflow_status}    \\
        Transitions stop-and-go     & {stopgo_r2:.3f}         & > 0.8          & {stopgo_status}      \\
        Interactions motos-voitures & {interaction_r2:.3f}    & > 0.8          & {interaction_status} \\
        \hline
    \end{tabular}
\end{table}

\paragraph{Validation Temporelle}

Capacité prédictive sur différents horizons temporels :

\begin{table}[H]
    \centering
    \caption{Précision prédictive par horizon temporel}
    \label{tab:temporal_validation}
    \begin{tabular}{|c|c|c|c|c|}
        \hline
        \textbf{Horizon (min)} & \textbf{MAPE (\%)} & \textbf{Theil U}  & \textbf{Seuil MAPE} & \textbf{Statut} \\
        \hline
        5                      & {mape_5min:.2f}    & {theil_5min:.3f}  & < 10\%              & {status_5min}   \\
        15                     & {mape_15min:.2f}   & {theil_15min:.3f} & < 15\%              & {status_15min}  \\
        30                     & {mape_30min:.2f}   & {theil_30min:.3f} & < 20\%              & {status_30min}  \\
        60                     & {mape_60min:.2f}   & {theil_60min:.3f} & < 25\%              & {status_60min}  \\
        \hline
    \end{tabular}
\end{table}

\subsubsection{Tests de Robustesse}

\paragraph{Conditions Dégradées}

Validation de la stabilité sous conditions de qualité routière dégradée :

\begin{table}[H]
    \centering
    \caption{Performance sous conditions dégradées}
    \label{tab:robustness_validation}
    \begin{tabular}{|l|c|c|c|c|}
        \hline
        \textbf{Condition R(x)}   & \textbf{MAPE (\%)}     & \textbf{Stabilité}          & \textbf{Seuil} & \textbf{Statut}      \\
        \hline
        R(x) = 1.0 (normal)       & {mape_normal:.2f}      & {stability_normal:.3f}      & < 15\%         & {status_normal}      \\
        R(x) = 0.7 (dégradé)      & {mape_degraded_07:.2f} & {stability_degraded_07:.3f} & < 20\%         & {status_degraded_07} \\
        R(x) = 0.5 (très dégradé) & {mape_degraded_05:.2f} & {stability_degraded_05:.3f} & < 25\%         & {status_degraded_05} \\
        R(x) variable spatial     & {mape_variable:.2f}    & {stability_variable:.3f}    & < 18\%         & {status_variable}    \\
        \hline
    \end{tabular}
\end{table}

\paragraph{Perturbations Externes}

Réponse aux événements exceptionnels :

\begin{itemize}
    \item Fermeture voie temporaire : Temps adaptation = {adaptation_time_closure:.1f} min
    \item Variation débit d'entrée +50\% : Stabilisation en {stabilization_time_increase:.1f} min
    \item Incident ponctuel : Propagation conforme observations (R² = {incident_propagation_r2:.3f})
\end{itemize}

\textbf{Résultat} : La validation comportementale et la robustesse confirment les \textbf{revendications R3 et R4} sur la fidélité du jumeau numérique et la conservation des propriétés physiques.

% Figure placeholder for digital twin validation
\begin{figure}[H]
    \centering
    \includegraphics[width=0.9\textwidth]{{{digital_twin_validation_plot_path}}}
    \caption{Validation comportementale du jumeau numérique : reproduction des patterns observés}
    \label{fig:digital_twin_validation}
\end{figure}