% Template pour Section 7.4 - Calibration Données Réelles
% Auto-generated by validation framework

\subsection{Validation par Calibration sur Données Réelles}

Cette section présente la calibration et validation du modèle ARZ sur les données de trafic réel de Victoria Island.

\subsubsection{Processus de Calibration}

Les paramètres du modèle ont été calibrés par optimisation sur les données observées :

\begin{table}[H]
    \centering
    \caption{Paramètres calibrés vs valeurs par défaut}
    \label{tab:calibrated_parameters}
    \begin{tabular}{|l|c|c|c|c|}
        \hline
        \textbf{Paramètre} & \textbf{Valeur initiale} & \textbf{Valeur calibrée}  & \textbf{Unité} & \textbf{Variation}          \\
        \hline
        $V_0$ motos        & {V0_moto_initial:.2f}    & {V0_moto_calibrated:.2f}  & km/h           & {V0_moto_variation:+.1f}\%  \\
        $V_0$ voitures     & {V0_car_initial:.2f}     & {V0_car_calibrated:.2f}   & km/h           & {V0_car_variation:+.1f}\%   \\
        $\tau$ motos       & {tau_moto_initial:.2f}   & {tau_moto_calibrated:.2f} & s              & {tau_moto_variation:+.1f}\% \\
        $\tau$ voitures    & {tau_car_initial:.2f}    & {tau_car_calibrated:.2f}  & s              & {tau_car_variation:+.1f}\%  \\
        $\rho_{max}$       & {rho_max_initial:.3f}    & {rho_max_calibrated:.3f}  & veh/m          & {rho_max_variation:+.1f}\%  \\
        \hline
    \end{tabular}
\end{table}

\subsubsection{Métriques de Validation}

\paragraph{Analyse MAPE (Mean Absolute Percentage Error)}

L'erreur relative moyenne absolue sur les flux agrégés respecte le seuil industrie :

\begin{table}[H]
    \centering
    \caption{Métriques MAPE par segment}
    \label{tab:mape_validation}
    \begin{tabular}{|l|c|c|c|c|}
        \hline
        \textbf{Segment}         & \textbf{MAPE (\%)}   & \textbf{Seuil} & \textbf{Statut}      & \textbf{N observations} \\
        \hline
        {segment_1_name}         & {segment_1_mape:.2f} & < 15\%         & {segment_1_status}   & {segment_1_n_obs}       \\
        {segment_2_name}         & {segment_2_mape:.2f} & < 15\%         & {segment_2_status}   & {segment_2_n_obs}       \\
        {segment_3_name}         & {segment_3_mape:.2f} & < 15\%         & {segment_3_status}   & {segment_3_n_obs}       \\
        {segment_4_name}         & {segment_4_mape:.2f} & < 15\%         & {segment_4_status}   & {segment_4_n_obs}       \\
        {segment_5_name}         & {segment_5_mape:.2f} & < 15\%         & {segment_5_status}   & {segment_5_n_obs}       \\
        \hline
        \textbf{Moyenne globale} & {global_mape:.2f}    & < 15\%         & {global_mape_status} & {total_observations}    \\
        \hline
    \end{tabular}
\end{table}

\paragraph{Statistique GEH (Geoffrey E. Havers)}

La validation selon le standard GEH pour les flux de trafic :

\begin{table}[H]
    \centering
    \caption{Validation GEH par lien routier}
    \label{tab:geh_validation}
    \begin{tabular}{|l|c|c|c|c|}
        \hline
        \textbf{Lien routier} & \textbf{GEH}     & \textbf{Seuil} & \textbf{Statut} & \textbf{Flux obs./sim.}              \\
        \hline
        {link_1_name}         & {link_1_geh:.2f} & < 5.0          & {link_1_status} & {link_1_observed}/{link_1_simulated} \\
        {link_2_name}         & {link_2_geh:.2f} & < 5.0          & {link_2_status} & {link_2_observed}/{link_2_simulated} \\
        {link_3_name}         & {link_3_geh:.2f} & < 5.0          & {link_3_status} & {link_3_observed}/{link_3_simulated} \\
        {link_4_name}         & {link_4_geh:.2f} & < 5.0          & {link_4_status} & {link_4_observed}/{link_4_simulated} \\
        {link_5_name}         & {link_5_geh:.2f} & < 5.0          & {link_5_status} & {link_5_observed}/{link_5_simulated} \\
        \hline
    \end{tabular}
\end{table}

\textbf{Taux d'acceptation GEH} : {geh_acceptance_rate:.1f}\% des liens respectent GEH < 5 (seuil industrie : 85\%)

\subsubsection{Coefficient de Theil}

L'évaluation de la qualité prédictive globale du modèle :

\begin{itemize}
    \item Coefficient de Theil U = {theil_coefficient:.3f} (seuil < 0.3 : {theil_status})
    \item RMSE normalisé = {normalized_rmse:.3f}
    \item Coefficient de corrélation R² = {r_squared:.3f}
\end{itemize}

\textbf{Résultat} : Les métriques de calibration valident la \textbf{revendication R2} sur la reproduction fidèle des données de trafic observées.

% Figure placeholder for calibration results
\begin{figure}[H]
    \centering
    \includegraphics[width=0.9\textwidth]{{{calibration_comparison_plot_path}}}
    \caption{Comparaison flux observés vs simulés après calibration}
    \label{fig:calibration_validation}
\end{figure}