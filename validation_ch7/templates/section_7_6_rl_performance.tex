% Template pour Section 7.6 - Performance Agents RL
% Auto-generated by validation framework

\subsection{Validation des Performances des Agents RL}

Cette section évalue les performances des agents d'apprentissage par renforcement dans le contrôle du trafic et valide leur intégration avec le simulateur ARZ.

\subsubsection{Validation de l'Interface ARZ-RL}

\paragraph{Tests de Couplage}

Validation de la stabilité et de la cohérence de l'interface de couplage :

\begin{table}[H]
    \centering
    \caption{Métriques de couplage ARZ-RL}
    \label{tab:rl_coupling}
    \begin{tabular}{|l|c|c|c|c|}
        \hline
        \textbf{Métrique}          & \textbf{Valeur mesurée}   & \textbf{Seuil} & \textbf{Statut}              & \textbf{Unité} \\
        \hline
        Latence communication      & {coupling_latency:.2f}    & < 10           & {coupling_latency_status}    & ms             \\
        Synchronisation temporelle & {time_sync_error:.2e}     & < 1e-3         & {time_sync_status}           & s              \\
        Perte d'information        & {information_loss:.2f}    & < 1\%          & {information_loss_status}    & \%             \\
        Stabilité numérique        & {numerical_stability:.3f} & > 0.99         & {numerical_stability_status} & -              \\
        \hline
    \end{tabular}
\end{table}

\paragraph{Cohérence des Observations}

Validation de la cohérence des états observés par l'agent RL :

\begin{itemize}
    \item Densités observées : Écart moyen < {density_observation_error:.2e}
    \item Vitesses moyennes : Précision ± {velocity_observation_precision:.2f} km/h
    \item Flux aux capteurs : Corrélation R² = {flow_observation_correlation:.3f}
\end{itemize}

\subsubsection{Évaluation des Performances}

\paragraph{Métriques de Performance vs Baseline}

Comparaison des performances RL par rapport au contrôle témoin sans optimisation :

\begin{table}[H]
    \centering
    \caption{Performance agents RL vs baseline témoin}
    \label{tab:rl_performance}
    \begin{tabular}{|l|c|c|c|c|}
        \hline
        \textbf{Métrique}        & \textbf{Baseline}          & \textbf{Agent RL}    & \textbf{Amélioration}            & \textbf{Statut}      \\
        \hline
        Temps trajet moyen (min) & {baseline_travel_time:.2f} & {rl_travel_time:.2f} & {travel_time_improvement:+.1f}\% & {travel_time_status} \\
        Vitesse moyenne (km/h)   & {baseline_avg_speed:.2f}   & {rl_avg_speed:.2f}   & {avg_speed_improvement:+.1f}\%   & {avg_speed_status}   \\
        Débit maximal (veh/h)    & {baseline_max_flow:.0f}    & {rl_max_flow:.0f}    & {max_flow_improvement:+.1f}\%    & {max_flow_status}    \\
        Temps arrêt total (h)    & {baseline_stop_time:.2f}   & {rl_stop_time:.2f}   & {stop_time_improvement:+.1f}\%   & {stop_time_status}   \\
        Consommation carburant   & {baseline_fuel:.2f}        & {rl_fuel:.2f}        & {fuel_improvement:+.1f}\%        & {fuel_status}        \\
        \hline
    \end{tabular}
\end{table}

\paragraph{Stabilité de l'Apprentissage}

Validation de la convergence et de la reproductibilité de l'apprentissage :

\begin{table}[H]
    \centering
    \caption{Stabilité apprentissage RL (multiple seeds)}
    \label{tab:rl_stability}
    \begin{tabular}{|c|c|c|c|c|}
        \hline
        \textbf{Seed}       & \textbf{Récompense finale} & \textbf{Convergence (épisodes)} & \textbf{Variance}     & \textbf{Statut}            \\
        \hline
        1                   & {reward_seed_1:.2f}        & {convergence_seed_1}            & {variance_seed_1:.3f} & {status_seed_1}            \\
        2                   & {reward_seed_2:.2f}        & {convergence_seed_2}            & {variance_seed_2:.3f} & {status_seed_2}            \\
        3                   & {reward_seed_3:.2f}        & {convergence_seed_3}            & {variance_seed_3:.3f} & {status_seed_3}            \\
        4                   & {reward_seed_4:.2f}        & {convergence_seed_4}            & {variance_seed_4:.3f} & {status_seed_4}            \\
        5                   & {reward_seed_5:.2f}        & {convergence_seed_5}            & {variance_seed_5:.3f} & {status_seed_5}            \\
        \hline
        \textbf{Moyenne}    & {reward_mean:.2f}          & {convergence_mean:.0f}          & {variance_mean:.3f}   & {overall_stability_status} \\
        \textbf{Écart-type} & {reward_std:.3f}           & {convergence_std:.0f}           & {variance_std:.3f}    & -                          \\
        \hline
    \end{tabular}
\end{table}

\subsubsection{Analyse Comportementale}

\paragraph{Stratégies Apprises}

Caractérisation des stratégies de contrôle développées par l'agent :

\begin{itemize}
    \item \textbf{Gestion proactive} : Anticipation congestion {proactive_anticipation:.1f} sec en avance
    \item \textbf{Coordination feux} : Synchronisation {coordination_efficiency:.1f}\% plus efficace
    \item \textbf{Adaptation dynamique} : Réponse aux perturbations en {adaptation_response_time:.1f} sec
    \item \textbf{Optimisation multi-classes} : Équilibrage motos/voitures (ratio {class_balance_ratio:.2f})
\end{itemize}

\paragraph{Robustesse des Politiques}

Test de la robustesse des politiques apprises sous conditions variées :

\begin{table}[H]
    \centering
    \caption{Robustesse des politiques RL}
    \label{tab:rl_robustness}
    \begin{tabular}{|l|c|c|c|c|}
        \hline
        \textbf{Condition de test} & \textbf{Performance maintenue}  & \textbf{Seuil} & \textbf{Statut}       & \textbf{Commentaire}   \\
        \hline
        Débit +25\%                & {robustness_flow_25:.1f}\%      & > 80\%         & {status_flow_25}      & {comment_flow_25}      \\
        Débit -25\%                & {robustness_flow_minus25:.1f}\% & > 80\%         & {status_flow_minus25} & {comment_flow_minus25} \\
        Incident voie fermée       & {robustness_incident:.1f}\%     & > 70\%         & {status_incident}     & {comment_incident}     \\
        Conditions météo           & {robustness_weather:.1f}\%      & > 75\%         & {status_weather}      & {comment_weather}      \\
        Changement composition     & {robustness_composition:.1f}\%  & > 85\%         & {status_composition}  & {comment_composition}  \\
        \hline
    \end{tabular}
\end{table}

\subsubsection{Validation Multi-Scénarios}

Performance de l'agent RL sur différents scénarios de trafic :

\begin{table}[H]
    \centering
    \caption{Performance RL par type de scénario}
    \label{tab:rl_scenarios}
    \begin{tabular}{|l|c|c|c|c|}
        \hline
        \textbf{Type scénario} & \textbf{Amélioration temps}        & \textbf{Amélioration débit}        & \textbf{Score global} & \textbf{Statut}   \\
        \hline
        Trafic fluide          & {improvement_freeflow_time:+.1f}\% & {improvement_freeflow_flow:+.1f}\% & {score_freeflow:.2f}  & {status_freeflow} \\
        Congestion modérée     & {improvement_moderate_time:+.1f}\% & {improvement_moderate_flow:+.1f}\% & {score_moderate:.2f}  & {status_moderate} \\
        Congestion sévère      & {improvement_severe_time:+.1f}\%   & {improvement_severe_flow:+.1f}\%   & {score_severe:.2f}    & {status_severe}   \\
        Conditions mixtes      & {improvement_mixed_time:+.1f}\%    & {improvement_mixed_flow:+.1f}\%    & {score_mixed:.2f}     & {status_mixed}    \\
        \hline
    \end{tabular}
\end{table}

\textbf{Résultat} : Les métriques de performance et de robustesse valident la \textbf{revendication R5} sur l'efficacité du couplage ARZ-RL pour l'optimisation du trafic urbain.

% Figure placeholder for RL performance analysis
\begin{figure}[H]
    \centering
    \includegraphics[width=0.9\textwidth]{{{rl_performance_plot_path}}}
    \caption{Évolution des performances de l'agent RL durant l'apprentissage}
    \label{fig:rl_performance_evolution}
\end{figure}