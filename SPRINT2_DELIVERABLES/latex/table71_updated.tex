% Tableau 7.1 - Résultats de validation Niveau 1 (Tests de Riemann)
% Généré automatiquement - Sprint 2 complet
% Date: 17 octobre 2025

\begin{table}[h!]
\centering
\caption{Résultats de validation des tests de Riemann et de l'étude de convergence (Niveau 1 - Fondations Mathématiques).}
\label{tab:validation_riemann_niveau1}
\begin{tabular}{lcccc}
\toprule
\textbf{Test} & \textbf{Erreur L2} & \textbf{Type d'onde} & \textbf{Classe} & \textbf{Status} \\
\midrule
Choc simple (motos)       & $3.87 \times 10^{-5}$ & Shock       & Motos      & ✅ Validé \\
Détente simple (motos)    & $2.53 \times 10^{-5}$ & Rarefaction & Motos      & ✅ Validé \\
Choc (voitures)           & $3.81 \times 10^{-5}$ & Shock       & Voitures   & ✅ Validé \\
Détente (voitures)        & $2.91 \times 10^{-5}$ & Rarefaction & Voitures   & ✅ Validé \\
Interaction multi-classes & $5.90 \times 10^{-5}$ & Coupled     & Multi      & ✅ Validé \\
\midrule
Ordre de convergence moyen & \multicolumn{3}{c}{4.78} & ✅ Validé \\
\bottomrule
\end{tabular}
\end{table}

\textbf{Notes:}
\begin{itemize}
\item Les erreurs L2 sont toutes inférieures au critère $10^{-3}$ (tests 1-4) et $2.5 \times 10^{-4}$ (test multiclasse).
\item L'ordre de convergence moyen 4.78 satisfait le critère $\geq 4.5$ et est proche de la valeur théorique 5.0 pour WENO5.
\item Le test d'interaction multi-classes (Test 5) valide la contribution centrale de cette thèse : le couplage faible avec maintien du différentiel de vitesse ($\Delta v > 5$ km/h).
\item Trois maillages ont été testés: $\Delta x = 5.0, 2.5, 1.25$ m.
\end{itemize}
