\chapter{Discussion Générale et Perspectives}
\label{chap:discussion_perspectives}

\section{Introduction}
% Contenu de l'introduction

\section{Synthèse des Principaux Résultats}
\subsection{Contributions Méthodologiques}
% Contenu de la sous-section

\subsection{Contributions Techniques}
% Contenu de la sous-section

\subsection{Validation Expérimentale}
% Contenu de la sous-section

\section{Discussion Critique des Résultats}
\subsection{Forces de l'Approche Globale}
% Contenu de la sous-section

\subsection{Faiblesses et Limitations Identifiées}
% Contenu de la sous-section

\section{Généralisabilité Régionale et Perspectives de Déploiement}
\subsection{Applicabilité aux Autres Villes Ouest-Africaines}
% Contenu de la sous-section

\subsection{Stratégie de Déploiement : De Lagos vers Cotonou}
% Contenu de la sous-section

% NOTE ÉDITORIALE : Cette section doit faire référence à l'analyse d'impact sociétal et économique
% développée dans la section \ref{subsec:impact_socioeconomique} du chapitre 7, particulièrement
% pour justifier les priorités de déploiement et les modèles de financement basés sur le ROI démontré.

\subsection{Besoins en Données Locales pour l'Adaptation}
% Contenu de la sous-section

\section{Axes de Recherche Futurs}
\subsection{Extensions du Modèle}
% Contenu de la sous-section

\subsection{Améliorations de l'IA}
% Contenu de la sous-section

\subsection{Validation Étendue}
% Contenu de la sous-section

\section{Limites et Remarques Finales}
% Contenu de la section