\chapter{Validation Intégrée du Modèle ARZ Étendu, du Jumeau Numérique et de l'Agent d'Apprentissage par Renforcement}
\label{chap:validation_entrainement}

\section{Introduction et Logique de Validation}
\label{sec:intro_logique_validation}

Ce chapitre constitue l'aboutissement logique de notre démarche de recherche. Après avoir développé un modèle ARZ étendu multi-classes (Chapitres~\ref{chap:formulation_modele} et~\ref{chap:fondements_mathematiques}), implémenté une stratégie de résolution numérique haute-fidélité (Chapitre~\ref{chap:strategie_resolution_numerique}), construit un jumeau numérique du corridor de Victoria Island (Chapitre~\ref{chap:construction_jn}), et conçu un environnement d'apprentissage par renforcement standardisé (Chapitre~\ref{chap:conception_env_rl}), il est temps de valider rigoureusement chacune de ces contributions.

\subsection{Fil Conducteur : Du Segment au Réseau, du Modèle au Contrôle}
\label{subsec:fil_conducteur}

Notre approche de validation suit une progression méthodique du simple au complexe :
\begin{enumerate}
    \item \textbf{Validation physique fondamentale} : Vérification que le modèle ARZ étendu reproduit correctement les phénomènes physiques attendus sur des segments isolés.
    \item \textbf{Validation des couplages} : Test de la cohérence aux jonctions et intersections, particulièrement pour les carrefours à feux.
    \item \textbf{Validation numérique} : Confirmation de la précision et de la stabilité de la méthode de résolution.
    \item \textbf{Validation du jumeau numérique} : Calibration et confrontation avec les données réelles du corridor de Victoria Island.
    \item \textbf{Validation de l'environnement RL} : Vérification de la cohérence du MDP et des métriques de performance.
    \item \textbf{Validation de l'apprentissage} : Entraînement des agents et comparaison avec les méthodes de référence.
    \item \textbf{Tests de scénarios et robustesse} : Évaluation dans des conditions dégradées et des situations extrêmes.
\end{enumerate}

\subsection{Hypothèses et Revendications Clés}
\label{subsec:hypotheses_revendications}

Nos travaux reposent sur six revendications principales (R1-R6) que ce chapitre se propose de valider :

\begin{itemize}
    \item \textbf{R1} : Le modèle ARZ étendu multi-classes capture fidèlement les spécificités comportementales du trafic ouest-africain (gap-filling, interweaving, creeping des motos).
    \item \textbf{R2} : La prise en compte de la qualité d'infrastructure R(x) améliore significativement la précision du modèle.
    \item \textbf{R3} : La stratégie numérique FVM + WENO garantit une résolution stable et précise du système hyperbolique couplé.
    \item \textbf{R4} : Le jumeau numérique du corridor de Victoria Island reproduit les conditions de trafic réelles avec une précision acceptable pour l'optimisation.
    \item \textbf{R5} : L'agent d'apprentissage par renforcement surpasse les méthodes de contrôle traditionnelles (plans fixes, contrôle actuated) sur les métriques opérationnelles clés.
    \item \textbf{R6} : La méthodologie de transposition régionale permet d'adapter les observations béninoises au contexte de Lagos avec succès.
\end{itemize}

% TODO: Ajouter une figure illustrant le pipeline de validation

\section{Cadre de Validation : Données, Métriques et Critères}
\label{sec:cadre_validation}

\subsection{Sources de Données}
\label{subsec:sources_donnees}

Notre validation repose sur plusieurs sources de données complémentaires :

\begin{itemize}
    \item \textbf{Données statiques} : Topologie du réseau extraite d'OpenStreetMap (OSM) avec enrichissement manuel des caractéristiques d'infrastructure.
    \item \textbf{Données dynamiques} : Vitesses de trafic et temps de parcours collectés via l'API TomTom Traffic sur 24h continues.
    \item \textbf{Données de référence} : Comptages de trafic disponibles dans la littérature \cite{ludi2020traffic} et observations comportementales documentées \cite{gomina2013urban}.
    \item \textbf{Données synthétiques} : Cas tests analytiques pour la validation des propriétés physiques et numériques.
\end{itemize}

\subsection{Métriques de Performance}
\label{subsec:metriques_performance}

Nous distinguons trois catégories de métriques selon le niveau de validation :

\subsubsection{Métriques Physiques}
\begin{itemize}
    \item \textbf{Erreur absolue moyenne (MAE)} et \textbf{erreur relative moyenne (MAPE)} sur les vitesses
    \item \textbf{Erreur quadratique moyenne (RMSE)} sur les densités
    \item \textbf{Coefficient de Theil (U)} pour l'évaluation globale des séries temporelles
    \item \textbf{Statistique GEH} pour les flux de trafic
    \item \textbf{Vitesses d'onde} observées vs théoriques
\end{itemize}

\subsubsection{Métriques Opérationnelles}
\begin{itemize}
    \item \textbf{Temps de parcours moyens} par segment et pour l'ensemble du corridor
    \item \textbf{Délais moyens} aux intersections
    \item \textbf{Longueurs de files d'attente} maximales et moyennes
    \item \textbf{Nombre d'arrêts} par véhicule
    \item \textbf{Débit total} du réseau (véhicules/heure)
\end{itemize}

\subsubsection{Métriques d'Apprentissage par Renforcement}
\begin{itemize}
    \item \textbf{Récompense moyenne} et sa convergence
    \item \textbf{Stabilité} (variance inter-exécutions)
    \item \textbf{Robustesse} aux variations de conditions initiales
    \item \textbf{Respect des contraintes} de sécurité (temps verts minimaux, etc.)
\end{itemize}

\subsection{Critères d'Acceptation}
\label{subsec:criteres_acceptation}

\begin{table}[htbp]
    \centering
    \caption{Critères d'acceptation par niveau de validation}
    \label{tab:criteres_acceptation}
    \begin{tabular}{|l|l|c|}
        \hline
        \textbf{Niveau}               & \textbf{Métrique}                & \textbf{Seuil d'acceptation} \\
        \hline
        \multirow{3}{*}{Physique}     & MAPE vitesse                     & $< 15\%$                     \\
                                      & RMSE densité normalisée          & $< 0.2$                      \\
                                      & Vitesse d'onde (erreur relative) & $< 10\%$                     \\
        \hline
        \multirow{3}{*}{Opérationnel} & MAPE temps de parcours           & $< 20\%$                     \\
                                      & GEH flux                         & $< 5$ (85\% des mesures)     \\
                                      & Coefficient de Theil             & $< 0.3$                      \\
        \hline
        \multirow{2}{*}{RL}           & Amélioration vs baseline         & $> 10\%$                     \\
                                      & Stabilité (CV récompense)        & $< 0.1$                      \\
        \hline
    \end{tabular}
\end{table}

% TODO: Justifier le choix des seuils par référence à la littérature

\section{Validation du Modèle ARZ Étendu sur Segment}
\label{sec:validation_arz_segment}

\textbf{Revendication testée : R1 et R3 - Le modèle ARZ étendu capture les phénomènes physiques attendus et est résolu avec précision.}

Cette section valide les propriétés physiques fondamentales de notre modèle sur des cas tests analytiques avant son application au réseau complet.

\subsection{Tests Analytiques et Benchmarks}
\label{subsec:tests_analytiques}

\subsubsection{Validation des États d'Équilibre et du Flux Libre}
% TODO: Implémenter et présenter les résultats des tests de flux libre
% - Vérification que v = V_e(ρ) en régime permanent
% - Comparaison des diagrammes fondamentaux théoriques vs simulés
% - Test de la différenciation entre classes (motos vs voitures)

\subsubsection{Propagation des Ondes de Choc}
% TODO: Tests de Riemann pour validation des vitesses d'onde
% - Configuration upstream/downstream avec discontinuité initiale
% - Mesure des vitesses de propagation et comparaison avec les valeurs propres théoriques
% - Validation de la conservation de masse à travers les chocs

\subsubsection{Dynamiques Hors Équilibre et Hystérésis}
% TODO: Tests de relaxation et d'hystérésis
% - Trajectoires de la variable w lors de perturbations
% - Temps de retour à l'équilibre vs τ théorique
% - Cycles charge-décharge montrant l'hystérésis

\subsubsection{Interactions Multi-Classes}
% TODO: Tests spécifiques aux comportements motos/voitures
% - Propagation différentielle des perturbations entre classes
% - Validation des termes de couplage (gap-filling, interweaving)
% - Impact du paramètre α sur les interactions

\subsubsection{Influence de l'Infrastructure R(x)}
% TODO: Tests avec variations spatiales de R(x)
% - Réduction localisée de vitesse libre
% - Récupération progressive après dégradation
% - Cohérence avec observations terrain

\subsection{Convergence de Grille et Précision Numérique}
\label{subsec:convergence_grille}
% TODO: Étude de convergence h -> 0
% - Ordre observé vs ordre théorique du schéma
% - Absence d'oscillations spurieuses (TVD)
% - Tests sur maillages raffinés

\subsection{Analyse de Sensibilité Locale}
\label{subsec:sensibilite_locale}
% TODO: Sensibilité aux paramètres clés
% - Impact de τ_i sur la dynamique temporelle
% - Sensibilité de V_e,i sur les états d'équilibre
% - Robustesse aux variations de p'_i

\subsection{Résultats et Validation}
\label{subsec:resultats_segment}
% TODO: Synthèse des résultats avec figures et tableaux
% - Tableau récapitulatif des écarts observés vs seuils
% - Figures des trajectoires caractéristiques
% - Validation ou rejet de R1 et R3 (partie segment)

\textbf{Validation : [À COMPLÉTER avec les résultats] - Revendications R1 et R3 acceptées/rejetées sur la base des preuves présentées dans les figures X.X et tableaux X.X.}

\section{Validation des Jonctions et Intersections}
\label{sec:validation_jonctions}

\textbf{Revendication testée : R1 et R3 - Les conditions de couplage aux nœuds préservent la cohérence physique.}

\subsection{Conservation de Masse aux Nœuds}
\label{subsec:conservation_masse_noeuds}
% TODO: Tests de conservation stricte
% - Configurations merge/diverge (2→1, 1→2)
% - Bilans de masse pour chaque classe
% - Tolérance numérique vs accumulation d'erreurs

\subsection{Cohérence de la Variable Lagrangienne}
\label{subsec:coherence_variable_w}
% TODO: Transmission de w à travers les jonctions
% - Continuité vs discontinuités physiques justifiées
% - Impact sur la dynamique post-jonction
% - Validation des règles de répartition

\subsection{Carrefours à Feux de Signalisation}
\label{subsec:carrefours_feux}
% TODO: Tests spécifiques aux intersections contrôlées
% - Respect des phases (rouge absolu)
% - Formation et écoulement des files
% - Débits de saturation observés vs théoriques
% - Transitions de phases

\subsection{Cas Limites et Robustesse}
\label{subsec:cas_limites_jonctions}
% TODO: Tests de robustesse
% - Situations de blocage (gridlock partiel)
% - Déséquilibres importants entre branches
% - Stabilité numérique aux jonctions

\subsection{Résultats et Validation}
\label{subsec:resultats_jonctions}
% TODO: Synthèse avec métriques de conservation et stabilité

\textbf{Validation : [À COMPLÉTER] - Revendication R1 et R3 (partie jonctions) acceptées/rejetées.}

\section{Validation de la Stratégie Numérique}
\label{sec:validation_numerique}

\textbf{Revendication testée : R3 - La méthode FVM + WENO garantit stabilité et précision.}

\subsection{Stabilité et Condition CFL}
\label{subsec:stabilite_cfl}
% TODO: Tests de stabilité
% - Respect de la condition CFL théorique
% - Comportement aux limites de stabilité
% - Impact du traitement des termes sources

\subsection{Schéma Bien Équilibré}
\label{subsec:schema_equilibre}
% TODO: Tests avec sources R(x)
% - Préservation des équilibres stationnaires
% - Absence de dérive numérique
% - Traitement des variations spatiales de R(x)

\subsection{Traitement de la Relaxation Raide}
\label{subsec:relaxation_raide}
% TODO: Tests avec τ petit
% - Splitting temporel vs IMEX
% - Absence de sur-amortissement
% - Stabilité pour τ → 0

\subsection{Analyse Précision/Coût Computationnel}
\label{subsec:precision_cout}
% TODO: Trade-off précision/temps calcul
% - Comparaison ordres WENO (3, 5)
% - Choix du flux numérique (HLL, HLLC)
% - Optimisation pour temps réel

\subsection{Résultats et Validation}
\label{subsec:resultats_numerique}
% TODO: Synthèse avec benchmarks de performance

\textbf{Validation : [À COMPLÉTER] - Revendication R3 acceptée/rejetée.}

\section{Calibration et Validation du Jumeau Numérique}
\label{sec:validation_jumeau_numerique}

\textbf{Revendication testée : R2, R4 et R6 - Le jumeau numérique reproduit fidèlement les conditions réelles du corridor de Victoria Island.}

\subsection{Stratégie de Calibration}
\label{subsec:strategie_calibration}
% TODO: Méthodologie de calibration détaillée
% - Paramètres à calibrer (V_{e,i}, τ_i, p'_i, R(x))
% - Fonction objective (combinaison métriques)
% - Algorithme d'optimisation employé
% - Fenêtres temporelles d'entraînement/validation

\subsection{Backtesting et Validation Croisée}
\label{subsec:backtesting}
% TODO: Tests temporels
% - Division données train/test (70%/30%)
% - Performance sur différentes périodes
% - Validation croisée temporelle

\subsection{Validation Spatio-Temporelle}
\label{subsec:validation_spatiotemporelle}
% TODO: Comparaison détaillée simulé vs observé
% - Heatmaps vitesse (espace-temps)
% - Profils de vitesse par segment
% - Évolution temporelle des temps de parcours
% - Métriques GEH et Theil par période

\subsubsection{Performance par Segment}
% TODO: Analyse segment par segment
% - Identification des segments problématiques
% - Corrélation avec caractéristiques d'infrastructure
% - Impact de la qualité des données TomTom

\subsubsection{Performance par Période}
% TODO: Analyse temporelle
% - Heures de pointe vs heures creuses
% - Week-end vs jours ouvrables
% - Événements particuliers identifiés

\subsection{Analyse des Sources d'Erreur}
\label{subsec:analyse_erreurs}
% TODO: Diagnostic des écarts
% - Erreurs liées aux données d'entrée
% - Limitations du modèle
% - Incertitudes de mesure TomTom
% - Phénomènes non modélisés


\subsection{Résultats et Validation}
\label{subsec:resultats_jumeau}
% TODO: Synthèse complète avec toutes les métriques
% - Tableau de performance par critère
% - Cartes de performance spatiale
% - Séries temporelles caractéristiques

\textbf{Validation : [À COMPLÉTER] - Revendications R2, R4 et R6 acceptées/rejetées avec preuves quantitatives.}

\section{Validation de l'Environnement d'Apprentissage par Renforcement}
\label{sec:validation_env_rl}

\textbf{Revendication testée : R5 (prérequis) - L'environnement MDP est cohérent et permet un apprentissage efficace.}

\subsection{Sanity Checks du MDP}
\label{subsec:sanity_checks_mdp}
% TODO: Vérifications de base
% - Bornes des espaces d'états et d'actions
% - Normalisation correcte des observations
% - Déterminisme et reproductibilité (seeds)
% - Cohérence récompense/objectifs

\subsection{Validation des Contraintes de Sécurité}
\label{subsec:contraintes_securite}
% TODO: Tests de respect des contraintes
% - Temps verts minimaux respectés
% - Durées maximales de cycles
% - Temps d'intergreen obligatoires
% - Gestion des phases interdites

\subsection{Analyse de la Fonction de Récompense}
\label{subsec:analyse_recompense}
% TODO: Validation de la fonction de récompense
% - Études d'ablation (composants individuels)
% - Corrélation avec métriques opérationnelles
% - Sensibilité aux poids relatifs
% - Évitement des optima locaux indésirables

\subsection{Benchmarks avec Baselines}
\label{subsec:benchmarks_baselines}
% TODO: Comparaison avec méthodes de référence
% - Plans fixes optimisés
% - Contrôle actuated/à seuils
% - Contrôle proportionnel simple
% - Validation des gains potentiels

\subsection{Résultats et Validation}
\label{subsec:resultats_env_rl}
% TODO: Synthèse de la validation environnement

\textbf{Validation : [À COMPLÉTER] - Prérequis pour R5 validés, environnement prêt pour l'entraînement.}

\section{Entraînement des Agents et Comparaison aux Baselines}
\label{sec:entrainement_agents}

\textbf{Revendication testée : R5 - L'agent RL surpasse les méthodes traditionnelles.}

\subsection{Protocole d'Entraînement}
\label{subsec:protocole_entrainement}
% TODO: Configuration expérimentale détaillée
% - Algorithmes testés (PPO, A2C, SAC)
% - Hyperparamètres et justifications
% - Architecture des réseaux de neurones
% - Horizon d'entraînement et early stopping
% - Nombre de seeds pour la robustesse statistique

\subsection{Courbes d'Apprentissage et Stabilité}
\label{subsec:courbes_apprentissage}
% TODO: Analyse de la convergence
% - Évolution de la récompense moyenne
% - Variance inter-seeds (boîtes à moustaches)
% - Détection de sur-apprentissage
% - Stabilité des politiques apprises

\subsection{Évaluation de Performance}
\label{subsec:evaluation_performance}
% TODO: Tests de performance détaillés
% - Métriques opérationnelles vs baselines
% - Tests de significativité statistique
% - Intervalles de confiance
% - Performance par période (pointe/creuse)

\subsubsection{Comparaison Quantitative}
% TODO: Tableaux de performance comparative
% - Délai moyen par véhicule
% - Débit total du réseau
% - Temps de parcours total
% - Nombre d'arrêts
% - Files d'attente maximales

\subsubsection{Analyse Temporelle}
% TODO: Performance dans le temps
% - Évolution sur une journée type
% - Robustesse aux variations de demande
% - Comportement en régime transitoire

\subsection{Robustesse et Généralisation}
\label{subsec:robustesse_generalisation}
% TODO: Tests de robustesse
% - Variations de conditions initiales
% - Changements de profils de demande
% - Dégradation des capteurs (bruit)
% - Transfer learning vers autres périodes

\subsection{Résultats et Validation}
\label{subsec:resultats_entrainement}
% TODO: Synthèse complète avec significativité statistique

\textbf{Validation : [À COMPLÉTER] - Revendication R5 acceptée/rejetée avec niveau de confiance statistique.}

\section{Tests de Scénarios et Analyse de Robustesse}
\label{sec:tests_scenarios}

\textbf{Revendication testée : R5 (robustesse) - Le système complet est resilient aux perturbations et conditions dégradées.}

\subsection{Perturbations de la Demande}
\label{subsec:perturbations_demande}
% TODO: Tests avec demande variable
% - Pics de trafic inattendus (+50%, +100%)
% - Événements ponctuels (manifestations, accidents)
% - Variations saisonnières simulées
% - Jours atypiques (fêtes, grèves)

\subsection{Dégradation de l'Infrastructure}
\label{subsec:degradation_infra}
% TODO: Tests avec infrastructure altérée
% - Modification de R(x) (chantiers, nids de poule)
% - Réduction temporaire de capacité
% - Conditions météorologiques adverses
% - Fermeture temporaire de voies

\subsection{Pannes et Dysfonctionnements}
\label{subsec:pannes_dysfonctionnements}
% TODO: Tests de contingence
% - Pannes de feux (mode clignotant)
% - Phases non fonctionnelles
% - Capteurs défaillants
% - Reconfiguration d'urgence

\subsection{Non-Conformité des Usagers}
\label{subsec:non_conformite_usagers}
% TODO: Tests avec comportements déviants
% - Non-respect des feux (pourcentage variable)
% - Stationnement gênant
% - Véhicules prioritaires (ambulances)
% - Comportements agressifs simulés

\subsection{Généralisation et Transfert}
\label{subsec:generalisation_transfert}
% TODO: Tests de transférabilité
% - Application à d'autres corridors (données limitées)
% - Transfer learning avec réentraînement minimal
% - Adaptation aux caractéristiques locales
% - Performance out-of-distribution

\subsection{Analyses Multi-Objectifs}
\label{subsec:analyses_multi_objectifs}
% TODO: Tests avec objectifs conflictuels (optionnel)
% - Trade-off délai/émissions
% - Équité entre usagers (piétons/véhicules)
% - Optimisation énergétique vs débit
% - Pareto-fronts des solutions

\subsection{Résultats et Validation}
\label{subsec:resultats_scenarios}
% TODO: Synthèse de la robustesse globale

\textbf{Validation : [À COMPLÉTER] - Robustesse du système validée dans X% des scénarios testés.}

\section{Analyses de Sensibilité et d'Incertitude}
\label{sec:analyses_sensibilite}

\subsection{Analyse de Sensibilité Globale}
\label{subsec:sensibilite_globale}
% TODO: Méthode de Sobol ou Morris screening
% - Paramètres les plus influents identifiés
% - Interactions entre paramètres
% - Seuils de sensibilité critique
% - Recommandations pour la calibration

\subsection{Propagation d'Incertitude}
\label{subsec:propagation_incertitude}
% TODO: Tests avec incertitudes réalistes
% - Bruits de mesure sur capteurs
% - Incertitudes sur paramètres calibrés
% - Données manquantes (interpolation)
% - Monte Carlo sur espaces d'entrée

\subsection{Identification des Points Critiques}
\label{subsec:points_critiques}
% TODO: Zones sensibles du système
% - Segments les plus critiques
% - Périodes de sensibilité maximale
% - Paramètres à surveiller prioritairement
% - Recommandations opérationnelles

\subsection{Résultats et Implications}
\label{subsec:resultats_sensibilite}
% TODO: Synthèse avec recommandations pratiques

\section{Limites, Validations Reportées et Travaux Futurs}
\label{sec:limites_travaux_futurs}

\subsection{Validations Non Réalisées}
\label{subsec:validations_non_realisees}
% TODO: Ce qui n'a pas pu être testé
% - Coordination multi-carrefours étendue (>3 intersections)
% - Validation avec données de comptage piétons réelles
% - Tests avec émissions mesurées (pas seulement proxies)
% - Validation sur longue période (>1 mois)
% - Comparaison avec système de contrôle existant in-situ

\subsection{Limitations Identifiées}
\label{subsec:limitations_identifiees}
% TODO: Limites actuelles reconnues
% - Dépendance aux données TomTom (qualité, couverture)
% - Modèle piétons/cyclistes simplifié
% - Absence de coordination inter-intersections optimale
% - Paramètres comportementaux moyennés (pas individualisés)
% - Coût computationnel pour déploiement temps réel

\subsection{Protocoles pour Validations Futures}
\label{subsec:protocoles_futurs}
% TODO: Recommandations pour la suite
% - Instrumentation terrain requise
% - Partenariats institutionnels nécessaires
% - Durées de tests recommandées
% - Méthodes de validation complémentaires
% - Risques associés et mitigation

\subsection{Extensions Recommandées}
\label{subsec:extensions_recommandees}
% TODO: Améliorations futures suggérées
% - Intégration de données IoT/capteurs dédiés
% - Machine Learning pour adaptation continue
% - Extension à réseaux multi-modaux
% - Optimisation multi-objectifs avancée
% - Interface opérateur temps réel

\section{Synthèse des Revendications Validées}
\label{sec:synthese_revendications}

\subsection{Récapitulatif des Preuves}
\label{subsec:recapitulatif_preuves}

\begin{table}[htbp]
    \centering
    \caption{Synthèse des revendications et de leur validation}
    \label{tab:synthese_revendications}
    \begin{tabular}{|l|l|c|l|}
        \hline
        \textbf{Revendication}  & \textbf{Métriques clés}           & \textbf{Statut} & \textbf{Preuves}   \\
        \hline
        R1: Modèle ARZ étendu   & MAPE vitesse, Conservation        & [TBD]           & Fig. X.X, Tab. X.X \\
        R2: Impact R(x)         & Amélioration RMSE                 & [TBD]           & Fig. X.X           \\
        R3: Stratégie numérique & Ordre convergence, Stabilité      & [TBD]           & Tab. X.X           \\
        R4: Jumeau numérique    & GEH, Theil U, MAPE TT             & [TBD]           & Fig. X.X-X.X       \\
        R5: Agent RL            & Amélioration vs baselines         & [TBD]           & Tab. X.X, Fig. X.X \\
        R6: Transposition       & Performance paramètres transposés & [TBD]           & Tab. X.X           \\
        \hline
    \end{tabular}
\end{table}

% TODO: Compléter le tableau avec les résultats réels

\subsection{Implications pour le Déploiement Réel}
\label{subsec:implications_deploiement}
% TODO: Conséquences pratiques des validations
% - Conditions de déploiement recommandées
% - Prérequis techniques et institutionnels
% - Bénéfices attendus quantifiés
% - Risques résiduels et leur gestion

\subsection{Impact sur la Transposition Régionale}
\label{subsec:impact_transposition_regionale}
% TODO: Généralisation des résultats
% - Applicabilité à d'autres villes ouest-africaines
% - Adaptations nécessaires par contexte
% - Méthodologie de transposition validée
% - Limites géographiques/culturelles identifiées

\section{Conclusion}
\label{sec:conclusion_validation}

Ce chapitre a présenté une validation systématique et rigoureuse de l'ensemble de notre chaîne de modélisation et d'optimisation. À travers une démarche progressive du segment isolé jusqu'au système complet en conditions réelles, nous avons testé et validé [X sur 6] de nos revendications principales.

% TODO: Conclusion synthétique avec bilan des validations
% - Résumé des principales validations réussies
% - Points de vigilance identifiés
% - Niveau de confiance global
% - Prêt pour déploiement/recommandations

Les résultats obtenus démontrent [À COMPLÉTER selon les résultats] et ouvrent la voie à une application pratique de notre approche dans les contextes urbains ouest-africains. Le chapitre suivant présentera les conclusions générales de ce travail et les perspectives de recherche qu'il ouvre.
