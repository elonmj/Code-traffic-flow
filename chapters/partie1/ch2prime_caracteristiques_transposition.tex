

\chapter{Caractéristiques du Trafic Urbain en Afrique de l'Ouest et Méthodologie de Transposition Régionale}
\label{chap:caracteristiques_transposition}

\section{Introduction}
La gestion du trafic urbain en Afrique de l'Ouest est confrontée à des défis majeurs en raison de l'urbanisation rapide, de l'augmentation du parc automobile et des infrastructures routières souvent inadéquates. Ce chapitre explore les caractéristiques spécifiques du trafic dans des villes comme Cotonou (Bénin) et Lagos (Nigeria), en mettant l'accent sur les similitudes et différences entre ces contextes. Il propose une méthodologie pour transposer les observations et modèles développés pour Cotonou vers le corridor de Victoria Island à Lagos, en tenant compte des spécificités régionales. Cette approche vise à optimiser la gestion du trafic dans des environnements complexes, en s'appuyant sur un jumeau numérique basé sur le modèle ARZ et un agent d'apprentissage par renforcement.

\section{Caractéristiques du Trafic Urbain en Afrique de l'Ouest}

\subsection{Contexte Régional Partagé}
Les villes d'Afrique de l'Ouest partagent des caractéristiques communes qui façonnent leurs dynamiques de trafic :
\begin{itemize}
    \item \textbf{Urbanisation rapide} : La croissance démographique et la migration vers les centres urbains, comme Cotonou et Lagos, augmentent la demande de mobilité \cite{kumar2011urban}.
    \item \textbf{Infrastructures routières limitées} : Les réseaux routiers sont souvent insuffisants, mal entretenus, ou inadaptés au volume de trafic, entraînant des embouteillages chroniques \cite{gomina2013urban}.
    \item \textbf{Dominance du transport informel} : Les motos et tricycles, comme les Zémidjans à Cotonou et les Okadas à Lagos, constituent une part majeure du transport, en raison de l'absence de systèmes de transport public formels robustes \cite{atakiti2016traffic}.
    \item \textbf{Comportements adaptatifs} : Les conducteurs adoptent des stratégies comme le contournement des obstacles ou l'utilisation de chemins informels pour naviguer dans des conditions complexes \cite{atakiti2016traffic}.
\end{itemize}
Ces similitudes offrent une base pour comparer et transposer les observations entre différents contextes urbains ouest-africains, tout en tenant compte des spécificités locales.

\subsection{Spécificités Observées au Bénin}
À Cotonou, le trafic est marqué par des caractéristiques distinctes :
\begin{itemize}
    \item \textbf{Prédominance des motos} : Les motos, notamment les Zémidjans, représentent 70 à 80 \% du trafic urbain, constituant environ 120 000 unités à Cotonou en 2017-2018 \cite{gomina2013urban}. Elles jouent un rôle clé dans le transport public informel, représentant 75 \% des déplacements internes.
    \item \textbf{Infrastructure hétérogène} : Le réseau routier est varié, avec seulement 28 \% des routes classées bitumées en 2008, bien que des efforts récents aient ajouté plus de 2 070 km de routes bitumées d'ici 2024 \cite{gomina2013urban}. Les routes en terre ou latérite, sensibles aux intempéries, dominent dans les zones périurbaines et rurales.
    \item \textbf{Comportements uniques} : Les motos adoptent des comportements spécifiques, tels que :
          \begin{itemize}
              \item \textit{Gap-filling} : Utilisation des espaces entre les véhicules pour maintenir une vitesse plus élevée, augmentant la densité de la chaussée.
              \item \textit{Interweaving} : Mouvements latéraux fréquents pour naviguer dans le trafic, optimisant le temps de trajet mais perturbant parfois les véhicules plus grands.
              \item \textit{Front-loading} : Positionnement en tête des files aux intersections, affectant la capacité et la dynamique des feux de signalisation.
              \item \textit{Creeping} : Progression lente dans les embouteillages, permettant aux motos de rester mobiles là où les voitures sont immobilisées.
          \end{itemize}
    \item \textbf{Régulation du trafic} : Les feux de signalisation sont rares et souvent défectueux en raison de problèmes d'entretien ou d'alimentation électrique. Les ronds-points, souvent congestionnés, et les agents de circulation humains introduisent une régulation informelle mais inconstante \cite{gomina2013urban}.
\end{itemize}
Ces caractéristiques nécessitent des modèles de trafic adaptés, comme une extension du modèle ARZ, pour capturer les interactions complexes entre motos, voitures et infrastructures.

\subsection{Choix du Corridor d'Application : Victoria Island (Lagos)}
Le corridor choisi pour l'application est l'axe formé par \textbf{Akin Adesola Street} et \textbf{Adeola Odeku Street} à Victoria Island, Lagos, long d'environ 2 à 3 km. Ce choix est motivé par plusieurs facteurs :
\begin{itemize}
    \item \textbf{Centre économique} : Victoria Island est un hub commercial et résidentiel, générant un trafic dense, comparable à Cotonou \cite{bbc2023lagos}.
    \item \textbf{Mixité des véhicules} : Le corridor présente une variété de véhicules, incluant voitures, bus, tricycles, et une forte proportion de motos (Okadas), similaire à Cotonou \cite{ludi2020traffic}.
    \item \textbf{Congestion chronique} : Les embouteillages sont fréquents, notamment aux heures de pointe, avec des temps de trajet pouvant atteindre 2 à 3 heures pour 41 km entre Ikorodu et Victoria Island \cite{bbc2023lagos}.
    \item \textbf{Intersections régulées} : Plusieurs intersections clés, comme celle d'Akin Adesola et Adeola Odeku, sont équipées de feux, idéales pour tester l'optimisation par apprentissage par renforcement \cite{ludi2020traffic}.
    \item \textbf{Infrastructure variable} : Les routes varient de bien entretenues à dégradées, permettant de calibrer le paramètre \(R(x)\) du modèle ARZ \cite{ludi2020traffic}.
\end{itemize}
Une étude de trafic à Victoria Island (Campos Square) montre que les piétons dominent, suivis des tricycles et motos, avec des volumes élevés sur des rues comme Igbosere et Tokunbo \cite{ludi2020traffic}. Le manque de parkings publics entraîne un stationnement sur la voie publique, exacerbant la congestion.

\begin{table}[htbp]
    \centering
    \caption{Comptages de trafic à Victoria Island (Campos Square), vendredi 7h30-8h30}
    \begin{tabular}{|l|c|c|c|c|c|c|c|c|c|}
        \hline
        Rue      & Voitures & Bus & Tricycles & Motos & Vélos & Remorques & Camions & BRT & Piétons \\
        \hline
        Igbosere & 244      & 106 & 682       & 314   & 2     & 1         & 5       & 1   & 390     \\
        Ajele    & 17       & 19  & 13        & 126   & 1     & 0         & 0       & 0   & 142     \\
        Bamgbose & 40       & 14  & 453       & 452   & 0     & 0         & 13      & 0   & 497     \\
        Tokunbo  & 30       & 9   & 194       & 365   & 1     & 0         & 1       & 0   & 514     \\
        \hline
    \end{tabular}
\end{table}

\subsection{Stratégie d'Adaptation Régionale}
Pour transposer les observations béninoises à Lagos, la stratégie repose sur :
\begin{itemize}
    \item \textbf{Identification des similitudes} : La dominance des motos et les conditions routières variables sont communes à Cotonou et Lagos, facilitant l'application des modèles \cite{kumar2011urban}.
    \item \textbf{Reconnaissance des différences} : Lagos est plus vaste, avec des infrastructures plus développées (par exemple, le métro Blue Line) et une régulation plus structurée via LASTMA \cite{lastma2020}.
    \item \textbf{Calibration des paramètres} : Ajuster les paramètres du modèle ARZ, comme \(V_{max,i}\) (vitesse maximale) et \(\tau_m\) (temps de réaction des motos), en utilisant des données locales de Lagos, telles que celles de TomTom ou Google Maps \cite{ludi2020traffic}.
\end{itemize}
Cette approche vise à valider l'applicabilité des modèles ARZ et des agents RL dans un contexte régional différent.

\section{Méthodologie de Transposition Régionale}

\subsection{Similitudes des Contextes Urbains Ouest-Africains}
Les contextes urbains ouest-africains partagent :
\begin{itemize}
    \item \textbf{Défis de mobilité} : Congestion chronique, manque d'infrastructures, et dépendance au transport informel, comme les motos et tricycles \cite{kumar2011urban}.
    \item \textbf{Comportements de conduite} : Les conducteurs adoptent des stratégies adaptatives, comme l'utilisation de chemins informels ou la priorité aux motos, influencées par des infrastructures variables \cite{atakiti2016traffic}.
    \item \textbf{Régulation faible} : Les environnements réglementaires sont souvent inconstants, avec une dépendance aux agents humains et une faible application des règles \cite{gomina2013urban}.
\end{itemize}
Ces similitudes justifient une méthodologie de transposition pour appliquer les modèles développés à Cotonou à d'autres contextes ouest-africains.

\subsection{Adaptation des Paramètres Comportementaux Béninois}
Pour transposer les paramètres comportementaux de Cotonou à Lagos :
\begin{itemize}
    \item \textbf{Collecte de données locales} : Observer les comportements des motos à Lagos (par exemple, \textit{gap-filling}, \textit{interweaving}), en s'appuyant sur des sources comme Google Maps ou des études locales \cite{ludi2020traffic}.
    \item \textbf{Ajustement du modèle ARZ} : Modifier les paramètres comme \(\alpha\) (densité perçue par les motos), \(V_{creeping}\) (vitesse de reptation), et \(\tau_m\) (temps de réaction) pour refléter les conditions de Lagos, en utilisant des données de trafic agrégées \cite{yoon2020design}.
    \item \textbf{Validation croisée} : Comparer les sorties du modèle avec des données réelles de Lagos, telles que les temps de trajet ou les niveaux de congestion, pour assurer la précision \cite{imoh2025analysis}.
\end{itemize}
Cette approche permet de capturer les nuances des dynamiques de trafic à Lagos tout en exploitant les connaissances acquises à Cotonou.

\subsection{Limites et Hypothèses de la Transposition}
La transposition repose sur plusieurs hypothèses et présente des limites :
\begin{itemize}
    \item \textbf{Hypothèses} : Les comportements de conduite (par exemple, \textit{gap-filling}, \textit{interweaving}) sont similaires entre Cotonou et Lagos ; les différences d'infrastructure peuvent être modélisées par des ajustements paramétriques \cite{kumar2011urban}.
    \item \textbf{Limites} : La rareté des données locales à Lagos, les différences d'échelle urbaine, et les variations dans l'application des régulations peuvent compliquer la transposition \cite{imoh2025analysis}. Une validation rigoureuse est nécessaire pour surmonter ces défis.
\end{itemize}

\section{Impact Économique de la Congestion et Justification de l'Optimisation}
\label{sec:impact_economique_justification}

La congestion urbaine en Afrique de l'Ouest génère des coûts économiques considérables qui justifient l'investissement dans des technologies d'optimisation du trafic. Cette section quantifie ces impacts pour établir le contexte économique de notre recherche.

\subsection{Coûts Macroéconomiques de la Congestion}

\subsubsection{Le Cas Emblématique de Lagos}
Lagos, principale métropole économique du Nigeria, illustre l'ampleur des coûts liés à la congestion urbaine :

\paragraph{Pertes Économiques Quantifiées}
Une étude du Danne Institute en partenariat avec la Financial Derivatives Company (2019) \cite{dannelagos2019} révèle des chiffres alarmants :
\begin{itemize}
    \item \textbf{Temps perdu individuel} : 3 heures productives perdues par jour et par Lagosien
    \item \textbf{Impact sur le PIB national} : N10,39 trillions (\$22,48 milliards USD) perdus annuellement
    \item \textbf{Manque à gagner fiscal} : N520,34 milliards de revenus internes non générés pour l'État de Lagos
    \item \textbf{Position mondiale} : Lagos détient l'indice de trafic le plus élevé au monde (348,69 selon Numbeo 2022)
\end{itemize}

\paragraph{Contexte de la Quatrième Économie Africaine}
Ces pertes représentent environ **6-8\% du PIB nigérian**, plaçant la congestion de Lagos parmi les freins économiques majeurs de l'Afrique de l'Ouest. Pour comparaison, Lagos concentre 40\% des véhicules nigérians sur moins de 1\% du territoire national.

\subsubsection{Extrapolation Régionale : Le Cas de Cotonou}
Bien que les données précises soient plus limitées pour Cotonou, les études de Fousseni et al. (2014) \cite{fousseni2014} permettent d'extrapoler :
\begin{itemize}
    \item \textbf{Échelle relative} : Population de 1,2 million vs. 21 millions à Lagos
    \item \textbf{Impact estimé} : 2-4\% du PIB béninois affecté par la congestion urbaine
    \item \textbf{Temps perdu} : 1,5-2 heures/jour/usager dans les zones denses
\end{itemize}

\subsection{Coûts Microéconomiques et Opérationnels}

\subsubsection{Impact sur les Opérateurs de Transport}
Les études récentes (ResearchGate, 2024) \cite{researchgate2024} documentent les coûts additionnels :
\begin{itemize}
    \item \textbf{Transport de fret} : N79 039 supplémentaires par an et par opérateur à Lagos
    \item \textbf{Transport public} : Surcoûts carburant de 15-25\% liés aux embouteillages
    \item \textbf{Usagers individuels} : N79 039 en frais de transport supplémentaires pour les utilisateurs de transport public
\end{itemize}

\subsubsection{Coûts d'Infrastructure et de Maintenance}
\begin{itemize}
    \item \textbf{Usure accélérée} : Les cycles arrêt/redémarrage augmentent l'usure véhiculaire
    \item \textbf{Vieillissement du parc} : Âge moyen >15 ans au Bénin, rendant la maintenance plus critique
    \item \textbf{Consommation énergétique} : Surconsommation de 20-30\% en conditions de stop-and-go
\end{itemize}

\subsection{Coûts Sociaux et Environnementaux}

\subsubsection{Impact Sanitaire}
\begin{itemize}
    \item \textbf{Pollution atmosphérique} : Concentration élevée de particules fines dans les corridors denses
    \item \textbf{Pathologies respiratoires} : Incidence accrue chez les conducteurs professionnels (zémidjan/okada)
    \item \textbf{Stress urbain} : Impact psychologique des temps de trajet prolongés
\end{itemize}

\subsubsection{Inégalités Sociales}
\begin{itemize}
    \item \textbf{Accessibilité différentielle} : Les populations à revenus modestes subissent davantage les retards
    \item \textbf{Secteur informel} : Impact disproportionné sur les travailleurs dépendants du transport public
    \item \textbf{Productivité urbaine} : Réduction de l'efficacité économique des centres-villes
\end{itemize}

\subsection{Justification Économique de l'Optimisation Technologique}

\subsubsection{Potentiel d'Amélioration}
Face à ces coûts considérables, l'optimisation technologique du contrôle de trafic présente un potentiel de retour sur investissement attractif :
\begin{itemize}
    \item \textbf{Coût relativement faible} : Technologies de capteurs et d'intelligence artificielle
    \item \textbf{Impact multiplicateur} : Une intersection optimisée bénéficie à des milliers d'usagers quotidiens
    \item \textbf{Évolutivité} : Solutions logicielles déployables sur l'infrastructure existante
\end{itemize}

\subsubsection{Contexte de Développement Technologique}
L'Afrique de l'Ouest, avec sa démographie jeune et sa croissance urbaine rapide, constitue un terrain d'application idéal pour :
\begin{itemize}
    \item \textbf{Innovation frugale} : Solutions adaptées aux contraintes locales
    \item \textbf{Leapfrogging technologique} : Adoption directe de technologies avancées
    \item \textbf{Impact social démontrable} : Bénéfices tangibles pour les populations urbaines
\end{itemize}

Cette analyse économique établit clairement la nécessité d'interventions technologiques pour réduire les coûts de la congestion. Elle justifie l'investissement en recherche et développement dans des solutions comme l'optimisation par apprentissage par renforcement, dont les bénéfices seront quantifiés au Chapitre~\ref{chap:evaluation_robustesse}.

\section{Conclusion du Chapitre}
Ce chapitre a analysé les caractéristiques du trafic urbain en Afrique de l'Ouest, quantifié l'impact économique de la congestion, et établi une méthodologie de transposition régionale. Les coûts considérables identifiés (N10,39 trillions/an à Lagos) justifient économiquement l'investissement dans des technologies d'optimisation. La méthodologie proposée permettra d'appliquer les observations béninoises au corridor de Victoria Island, ouvrant la voie au développement du jumeau numérique et de l'agent d'apprentissage par renforcement présentés dans les chapitres suivants.




