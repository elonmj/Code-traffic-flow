
\chapter{Extended Multiclass ARZ Model Formulation for Benin}
\label{chap:formulation_modele}

% Introduction au chapitre
Ce chapitre est dédié à la formulation mathématique détaillée du modèle macroscopique de trafic routier proposé pour le contexte béninois. Comme établi dans la revue de la littérature (Chapitre \ref{chap:revue_litterature}), les modèles de premier ordre comme le LWR sont insuffisants pour capturer la complexité dynamique observée, notamment les phénomènes hors équilibre et l'hétérogénéité marquée du parc de véhicules. Le modèle Aw-Rascle-Zhang (ARZ) \cite{AwKlarMaterneRascle2000, ZhangEtAl2003} a été identifié comme une base théorique plus appropriée en raison de ses propriétés mathématiques avantageuses et de sa capacité intrinsèque à modéliser l'anisotropie, l'hystérésis et les ondes \textit{stop-and-go} \cite{FanHertySeibold2014, yu2024traffic}.

L'objectif de ce chapitre est de construire une \textit{extension multi-classes} de ce cadre ARZ, spécifiquement conçue pour intégrer les caractéristiques distinctives du trafic au Bénin, telles que décrites au Chapitre \ref{chap:specificites_benin}. Cela inclut la prédominance écrasante des motocyclettes (Zémidjans), leurs comportements spécifiques (gap-filling, interweaving, creeping), et l'impact de la qualité variable de l'infrastructure routière.

\section{Base Multiclass ARZ Framework Selection}
\label{sec:base_multiclass_arz}

% Justification du multi-classe
La première étape cruciale dans la formulation du modèle est de reconnaître l'impératif d'une approche \textbf{multi-classes}. Le trafic au Bénin, comme démontré au Chapitre \ref{chap:specificites_benin}, est caractérisé par une \textbf{hétérogénéité extrême}, où les motocyclettes constituent la majorité écrasante du flux (souvent plus de 70-80\% en milieu urbain \cite{Djossou_ZemidjanCotonou}) et coexistent avec des voitures particulières, des camions, des bus et des tricycles. Les différences fondamentales en termes de taille, de capacités dynamiques (accélération, freinage), de manœuvrabilité, et surtout de comportements de conduite (voir Section \ref{sec:comportements_motos_impact}) entre les motos et les autres véhicules rendent tout modèle homogène (qui suppose un seul type de véhicule ou des comportements moyennés) intrinsèquement incapable de reproduire fidèlement la dynamique observée \cite{WongWong2002}. L'utilisation d'un modèle multi-classes permet de distinguer explicitement les propriétés et les interactions de différents groupes de véhicules.

% Choix du cadre ARZ multi-classe spécifique
Plusieurs approches existent pour étendre le modèle ARZ à un cadre multi-classes \cite{BenzoniGavageColombo2003, FanWork2015, ColomboMarcellini2020}. Pour ce travail, nous adoptons une formulation courante qui consiste à écrire un système d'équations ARZ pour chaque classe de véhicules, où les interactions entre les classes sont modélisées à travers les dépendances des fonctions clés (comme la vitesse d'équilibre et la fonction de pression) par rapport à l'état global du trafic (densités et/ou vitesses de toutes les classes).

Compte tenu de la dichotomie majeure observée au Bénin, nous considérerons \textbf{deux classes} principales :
\begin{itemize}
    \item Classe \( m \): Motocyclettes (incluant les Zémidjans)
    \item Classe \( c \): Autres véhicules (principalement voitures particulières, mais pouvant regrouper conceptuellement les véhicules plus larges et moins agiles)
\end{itemize}
Cette simplification permet de se concentrer sur l'interaction fondamentale moto-voiture, tout en gardant la possibilité d'ajouter d'autres classes dans des travaux futurs.

% Equations de base (avant extensions spécifiques)
Le système de base ARZ multi-classes, avant l'intégration des spécificités béninoises, s'écrit sous la forme suivante, incluant un terme de relaxation vers une vitesse d'équilibre. Pour chaque classe \( i \in \{m, c\} \):

\begin{align}
    \label{eq:arz_mass_conservation_i}
    \frac{\partial \rho_i}{\partial t} + \frac{\partial (\rho_i v_i)}{\partial x} &= 0 \\
    \label{eq:arz_momentum_relaxation_i}
    \frac{\partial w_i}{\partial t} + v_i \frac{\partial w_i}{\partial x} &= \frac{1}{\tau_i} (V_{e,i}(\rho_m, \rho_c) - v_i) \quad \text{avec} \quad w_i = v_i + p_i(\rho_m, \rho_c)
\end{align}

Où les variables sont définies comme suit : \( \rho_i \) (densité), \( v_i \) (vitesse), \( w_i \) (variable lagrangienne), \( p_i \) (pression), \( V_{e,i} \) (vitesse d'équilibre), et \( \tau_i \) (temps de relaxation).

Ce système (\ref{eq:arz_mass_conservation_i})-(\ref{eq:arz_momentum_relaxation_i}) forme le \textbf{squelette} de notre modèle. Les sections suivantes détailleront comment ce squelette est enrichi pour modéliser les comportements et contraintes spécifiques observés sur le terrain.

\section{Modeling Road Pavement Effects on Equilibrium Speed}
\label{sec:modeling_pavement}

L'une des caractéristiques marquantes du réseau routier béninois est la grande \textbf{variabilité de l'état du revêtement}. Pour intégrer cet effet, nous rendons la fonction de vitesse d'équilibre \( V_{e,i} \) explicitement dépendante d'un indicateur de qualité de la route, \( R(x) \). Cet indicateur module la vitesse maximale en flux libre \( V_{max,i} \), qui devient une fonction \( V_{max,i}(R(x)) \).
\begin{equation}
    \label{eq:Ve_depends_on_R}
    V_{e,i}(\rho_m, \rho_c, R(x)) = V_{max,i}(R(x)) \cdot f_i(\rho_m, \rho_c)
\end{equation}
où \( f_i \) est une fonction décroissante de la densité. Crucialement, les \textbf{motocyclettes (classe \( m \)) sont moins sensibles à la dégradation de la chaussée}. Le modèle capture ce comportement différentiel en s'assurant que la fonction \( V_{max,m}(R(x)) \) est moins affectée par une dégradation de \( R(x) \) que la fonction \( V_{max,c}(R(x)) \). L'introduction de cette dépendance spatiale rend le système non-homogène et a des implications directes sur la résolution numérique (Chapitre \ref{chap:analyse_numerique}).

\section{Modeling Motorcycle Gap-Filling: The Foundational Mechanism}
\label{sec:modeling_gap_filling}

Le \textbf{"gap-filling" (remplissage d'interstices)} est la capacité des motos à exploiter les espaces pour progresser en trafic dense. C'est le comportement fondamental qui sous-tend toutes les autres spécificités des motos. Dans le cadre ARZ, cette perception de l'espace est capturée par la \textbf{fonction de pression \( p_i \)}, qui représente la "gêne" ressentie par un conducteur.

Nous modélisons le gap-filling en introduisant le concept de \textbf{densité effective perçue}, \( \rho_{eff,i} \). Tandis que les voitures perçoivent la densité totale, les motos perçoivent une densité réduite grâce à leur capacité à ignorer une partie de l'encombrement créé par les voitures. Ceci est formalisé par un paramètre \( \alpha \in [0, 1) \):
\begin{equation}
    \label{eq:rho_eff_m}
    \rho_{eff,m} = \rho_m + \alpha \rho_c
\end{equation}
Un paramètre \( \alpha < 1 \) est le mécanisme clé : il signifie que la présence des voitures est perçue par les motos comme moins contraignante. Les fonctions de pression dépendent alors de cette perception différenciée :
\begin{align}
    p_m(\rho_m, \rho_c) &= P_m(\rho_{eff,m}) = P_m(\rho_m + \alpha \rho_c) \\
    p_c(\rho_m, \rho_c) &= P_c(\rho_m + \rho_c)
\end{align}
Ce mécanisme de perception réduite de la densité, formalisé par le paramètre `α`, constitue la pierre angulaire de notre modélisation du comportement des motos. Comme nous le verrons, il est non seulement la base du gap-filling, mais il justifie et permet également la modélisation de l'entrelacement et du creeping dans les sections suivantes.

\section{Modeling Motorcycle Interweaving: The Dynamic Action}
\label{sec:modeling_interweaving}

L'\textbf{entrelacement} (ou \textit{interweaving}) peut être vu comme la **manifestation dynamique du gap-filling**. Alors que le gap-filling décrit la *perception* de l'espace, l'entrelacement décrit l'*action* d'exploiter cet espace par des mouvements latéraux agiles. Dans notre modèle 1D, cet effet est capturé en renforçant les mécanismes déjà introduits et en y associant la notion de temps de réaction.

L'entrelacement est modélisé par la synergie de trois effets sur les fonctions du modèle :
1.  **Sur la Pression \( p_m \):** L'agilité de l'entrelacement renforce l'effet du gap-filling. Cela justifie une valeur faible pour le paramètre \( \alpha \) dans \( \rho_{eff,m} \), qui représente ainsi l'effet combiné de ces deux comportements.
2.  **Sur la Vitesse d'Équilibre \( V_{e,m} \):** L'entrelacement permet de maintenir une vitesse plus élevée en congestion, un effet qui sera quantifié plus précisément dans le cadre du "creeping" (Section \ref{sec:modeling_creeping}).
3.  **Sur le Temps de Relaxation \( \tau_m \):** C'est le mécanisme le plus direct pour modéliser l'agilité. L'entrelacement permet aux motocyclistes de réagir et d'adapter leur vitesse beaucoup plus rapidement. Ceci est modélisé en attribuant un **temps de relaxation \( \tau_m \) plus court** aux motos qu'aux voitures (\( \tau_m < \tau_c \)), et potentiellement en le rendant dépendant de la densité pour refléter une adaptation encore plus rapide lorsque le trafic ralentit.

Dans le cadre de cette thèse, nous privilégions cette approche qui modifie les fonctions \( p_i, V_{e,i}, \tau_i \) car elle conserve la structure mathématique du cadre ARZ. Les alternatives, comme l'ajout de termes d'interaction complexes, sont plus spéculatives et reléguées à des travaux futurs.

\section{Modeling Motorcycle "Creeping": The Ultimate Consequence}
\label{sec:modeling_creeping}

Le "creeping" (reptation) est la conséquence ultime et la plus visible de la capacité unique des motos à exploiter l'espace en conditions de congestion extrême, leur permettant de maintenir une vitesse faible mais non nulle. Ce comportement n'est possible que parce que le gap-filling et l'entrelacement leur permettent de percevoir un environnement moins contraint (`p_m`) et d'y réagir plus vite (`τ_m`).

Notre modèle intègre ce comportement final en modifiant la destination même de leur dynamique : la \textbf{vitesse d'équilibre \( V_{e,m} \)}. Alors que la vitesse des voitures tend vers zéro à la densité de bouchon, celle des motos tend vers une valeur positive, la vitesse de creeping \( V_{creeping} \)**.
\begin{equation}
    \label{eq:Ve_m_creeping}
    V_{e,m}(\rho_m, \rho_c, R(x)) = V_{creeping} + \left(V_{max,m}(R(x)) - V_{creeping}\right) \cdot g_m(\rho_m, \rho_c)
\end{equation}
où \( g_m \) est une fonction décroissante qui s'annule à la densité maximale. Pour les voitures, la même forme est utilisée avec \( V_{creeping} = 0 \).

Cette vitesse résiduelle n'est physiquement possible que parce que la perception de l'espace par les motos, modélisée via \( \rho_{eff,m} \), empêche leur fonction de pression \( p_m \) de diverger de la même manière que celle des voitures (\( p_c \)) lorsque la densité totale approche `ρ_jam`. Le modèle est ainsi **intrinsèquement cohérent** : la capacité à "ramper" est une conséquence directe de la perception différenciée de l'espace.

\section{Intersection Model: Source/Sink Terms and Coupling Conditions}
\label{sec:modeling_intersections}

La modélisation d'un réseau requiert la gestion des intersections. Un modèle de nœud doit assurer la conservation de la masse et appliquer des règles de distribution. Pour notre modèle de second ordre, le défi majeur est le couplage de la variable lagrangienne \( w_i \).

Étant donné la complexité mathématique des solveurs de Riemann aux jonctions pour notre modèle étendu, et pour conserver une approche numériquement robuste et tractable, **nous adoptons pour ce travail une condition de couplage simplifiée mais physiquement pertinente pour la variable \(w_i\)**. À chaque arc sortant, la valeur de \(w_i\) à l'entrée de l'arc est définie telle que la vitesse `v_i` correspond à la vitesse d'équilibre `Ve,i` calculée avec la nouvelle densité de flux sortant. Cette approche a l'avantage de garantir que les véhicules entrant dans un nouvel arc adoptent un comportement cohérent avec les conditions locales, évitant ainsi l'introduction d'états non-physiques. Les approches plus complexes basées sur des solveurs de Riemann complets sont laissées pour des travaux futurs.

\section{The Complete Extended ARZ Model Equations}
\label{sec:complete_model_equations}

Cette section synthétise les mécanismes de modélisation précédents en un système d'équations unifié. Ce système capture, dans un cadre mathématique cohérent, la synergie des comportements des motos et l'impact de l'infrastructure.

Le modèle étendu est un système de 4 EDP non linéaires couplées pour les variables d'état \(\rho_i(x, t)\) et \(w_i(x, t)\). Pour chaque classe \(i \in \{m, c\}\), les équations sont :
\begin{align}
    \label{eq:full_mass_i}
    \frac{\partial \rho_i}{\partial t} + \frac{\partial (\rho_i v_i)}{\partial x} &= 0 \\
    \label{eq:full_momentum_i}
    \frac{\partial w_i}{\partial t} + v_i \frac{\partial w_i}{\partial x} &= \frac{1}{\tau_i(\rho)} (V_{e,i}(\rho, R(x)) - v_i)
\end{align}
avec la relation fondamentale :
\begin{equation}
    \label{eq:full_vi_wi_pi}
    v_i = w_i - p_i(\rho_m, \rho_c)
\end{equation}

Les fonctions clés intègrent les spécificités du trafic béninois comme suit :

\paragraph{Fonctions de Pression \( p_i(\rho_m, \rho_c) \):}
Elles capturent la synergie du **gap-filling** et de **l'entrelacement** via la densité effective perçue.
\begin{align}
    p_m(\rho_m, \rho_c) &= P_m(\rho_m + \alpha \rho_c) \quad \text{avec } \alpha < 1 \\
    p_c(\rho_m, \rho_c) &= P_c(\rho_m + \rho_c)
\end{align}


\paragraph{Fonctions de Vitesse d'Équilibre \( V_{e,i}(\rho, R(x)) \):}
Dépendent de la densité totale \( \rho = \rho_m + \rho_c \) et de la qualité du revêtement \( R(x) \), intégrant les effets du revêtement (Section \ref{sec:modeling_pavement}) et du creeping pour les motos (Section \ref{sec:modeling_creeping}).
\begin{align}
    V_{e,m}(\rho, R(x)) &= V_{creeping} + \left(V_{max,m}(R(x)) - V_{creeping}\right) \cdot g_m(\rho) \\
    V_{e,c}(\rho, R(x)) &= V_{max,c}(R(x)) \cdot g_c(\rho)
\end{align}




où :
\begin{itemize}
    \item \( V_{creeping} \ge 0 \) est la vitesse résiduelle des motos en bouchon extrême (\( V_{creeping} > 0 \) pour modéliser le creeping).
    \item \( V_{max,i}(R(x)) \) est la vitesse maximale en flux libre pour la classe \( i \) sur un revêtement de qualité \( R(x) \). Ce sont des fonctions dépendant spatialement de \( R(x) \), où \( V_{max,m}(R) \) est calibré pour être moins sensible à la dégradation que \( V_{max,c}(R) \).
    \item \( g_m(\rho) \) et \( g_c(\rho) \) sont des fonctions décroissantes de la densité totale \( \rho \), telles que \( g_i(0)=1 \) et \( g_i(\rho_{jam})=0 \). La forme de \( g_m \) peut décroître moins rapidement que \( g_c \) pour refléter davantage l'avantage des motos. Une forme simple type Greenshields est \(g_i(\rho) = (1 - \rho / \rho_{jam})_+\), où \( (x)_+ = \max(x, 0) \).
    \item \( \rho_{jam} \) est la densité de bouchon physique maximale, une constante positive fixée.
    \item \( R(x) \) est un paramètre spatialement variable caractérisant la qualité de la route au point \( x \).
\end{itemize}

\begin{figure}[h!]
    \centering
    % Note : Remplacez 'images/conceptual_Ve.png' par le chemin réel de votre image.
    % Vous pouvez créer ce graphique simple avec Python/Matplotlib ou un autre outil.
    % \includegraphics[width=0.8\textwidth]{images/conceptual_Ve.png} 
    \fbox{\parbox[c][10cm][c]{0.8\textwidth}{\centering \large Placeholder for Conceptual Diagram \\ (Veuillez insérer votre graphique ici)}}
    \caption{Diagramme conceptuel des fonctions de vitesse d'équilibre \(V_{e,i}(\rho)\) pour les motos (classe \(m\), en bleu) et les voitures (classe \(c\), en rouge). La figure illustre trois effets clés modélisés : \textbf{(1)} une vitesse maximale \(V_{max,m}\) plus élevée pour les motos, \textbf{(2)} une décroissance plus lente de la vitesse des motos avec la densité (reflétant le gap-filling/interweaving), et \textbf{(3)} la vitesse de "creeping" \(V_{creeping} > 0\) pour les motos à la densité de bouchon \(\rho_{jam}\), alors que la vitesse des voitures s'annule.}
    \label{fig:conceptual_ve}
\end{figure}


\paragraph{Fonctions de Temps de Relaxation \( \tau_i(\rho) \):}
Elles capturent la différence de temps de réaction, principalement liée au comportement d'**entrelacement**. L'agilité des motos leur permet d'adapter leur vitesse plus rapidement aux conditions changeantes.
\begin{itemize}
    \item \( \tau_m(\rho) \): Représente le temps de relaxation court des motos, capturant leur capacité à réagir vivement. Pour ce travail, nous le considérons constant mais inférieur à celui des voitures.
    \item \( \tau_c(\rho) \): Représente le temps de relaxation plus long des voitures.
\end{itemize}
Nous posons donc \( \tau_m < \tau_c \). L'exploration de formes dépendant de la densité est une piste pour des raffinements futurs du modèle.

\paragraph{Conclusion sur la formulation du modèle}
Le système de 4 EDP couplées pour \((\rho_m, w_m, \rho_c, w_c)\), complété par la définition des fonctions \(p_i, V_{e,i}, \tau_i\), constitue notre **modèle étendu ARZ multi-classes sur un segment routier**. Il est fondamental de souligner que ce modèle n'est pas une simple collection de fonctionnalités, mais un système **cohérent et synergique**. La capacité à percevoir l'espace différemment (via `p_m`) permet une réaction plus rapide (via `τ_m`) et autorise un mouvement résiduel en congestion (via `Ve,m`).

Ce réalisme accru se traduit par un nombre significatif de paramètres (\(\alpha, V_{creeping}, \rho_{jam}\), ainsi que les paramètres définissant les fonctions \(P_i, V_{max,i}(R), g_i, \tau_i\)) dont l'estimation précise lors de la calibration (Chapitre \ref{chap:simulations_analyse}) sera une étape critique et déterminante pour la validité prédictive du modèle. La modélisation à l'échelle du réseau nécessitera de résoudre ce système sur chaque segment et d'appliquer les conditions de couplage aux nœuds (Section \ref{sec:modeling_intersections}). L'analyse mathématique et la résolution numérique de ce système complexe font l'objet des chapitres suivants.