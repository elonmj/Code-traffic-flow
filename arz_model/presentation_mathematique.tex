\documentclass{beamer}
\usepackage[utf8]{inputenc}
\usepackage[T1]{fontenc}
\usepackage{lmodern}
\usepackage{amsmath}
\usepackage{graphicx}

\usetheme{Madrid}
\title{La Chaîne Numérique d'un Modèle de Trafic Routier}
\subtitle{Du Problème Physique à la Simulation Numérique}
\author{Analyse du code source}
\date{\today}

\begin{document}

\frame{\titlepage}

%-------------------------------------------------
\section{Introduction : Le Problème}
%-------------------------------------------------

\begin{frame}
\frametitle{Le Défi : Modéliser le Trafic Routier}

\begin{columns}[T]
    \begin{column}{.5\textwidth}
        \textbf{Le monde réel :}
        \begin{itemize}
            \item Des millions de véhicules
            \item Comportements complexes (accélération, freinage)
            \item Interactions entre différents types de véhicules (voitures, motos)
            \item Formation d'embouteillages, ondes de choc
        \end{itemize}
        \pause
        \textbf{La question :}
        \begin{itemize}
            \item Comment prédire l'évolution du trafic ?
            \item Comment comprendre la formation des congestions ?
        \end{itemize}
    \end{column}
    \begin{column}{.5\textwidth}
        \includegraphics[width=\textwidth]{traffic_jam.jpg} % Placeholder image
        \tiny{Image conceptuelle}
    \end{column}
\end{columns}

\end{frame}

%-------------------------------------------------
\begin{frame}
\frametitle{L'Approche : Le Trafic comme un Fluide}

Au lieu de suivre chaque véhicule (approche \textbf{microscopique}), nous traitons le trafic comme un fluide continu (approche \textbf{macroscopique}).

\vspace{1cm}

\begin{columns}[T]
    \begin{column}{.5\textwidth}
        \textbf{Variables Clés :}
        \begin{itemize}
            \item $\rho(x, t)$: Densité de véhicules (veh/km)
            \item $v(x, t)$: Vitesse moyenne des véhicules (km/h)
        \end{itemize}
        \pause
        \textbf{L'Objectif :}
        \begin{itemize}
            \item Décrire l'évolution de $\rho$ et $v$ dans l'espace ($x$) et le temps ($t$) à l'aide d'équations mathématiques.
        \end{itemize}
    \end{column}
    \begin{column}{.5\textwidth}
        \includegraphics[width=\textwidth]{fluid_flow.jpg} % Placeholder image
        \tiny{Analogie avec un écoulement fluide}
    \end{column}
\end{columns}

\end{frame}

%-------------------------------------------------
\section{Le Modèle Mathématique (ARZ)}
%-------------------------------------------------

\begin{frame}
\frametitle{Le Modèle ARZ pour Deux Classes de Véhicules}

Le code implémente un modèle de type Aw-Rascle-Zhang (ARZ) pour deux classes : motos ($m$) et voitures ($c$).

\textbf{Le système d'Équations aux Dérivées Partielles (EDP) :}
\begin{align*}
    \frac{\partial \rho_m}{\partial t} + \frac{\partial (\rho_m v_m)}{\partial x} &= 0 \\
    \pause
    \frac{\partial \rho_c}{\partial t} + \frac{\partial (\rho_c v_c)}{\partial x} &= 0 \\
    \pause
    \frac{\partial w_m}{\partial t} + v_m \frac{\partial w_m}{\partial x} &= \frac{V_e(\rho_m, \rho_c) - v_m}{\tau_m} \\
    \pause
    \frac{\partial w_c}{\partial t} + v_c \frac{\partial w_c}{\partial x} &= \frac{V_e(\rho_m, \rho_c) - v_c}{\tau_c}
\end{align*}

\vspace{0.5cm}
\textbf{Variables :}
\begin{itemize}
    \item $\rho_m, \rho_c$: Densités des motos et voitures.
    \item $w_m, w_c$: Variables de "momentum" ou d'anticipation. $w = v + p(\rho)$.
    \item $v_m, v_c$: Vitesses réelles.
    \item $V_e$: Vitesse à l'équilibre (le "désir" des conducteurs).
    \item $\tau$: Temps de relaxation.
\end{itemize}

\end{frame}

%-------------------------------------------------
\begin{frame}
\frametitle{Comprendre les Équations (1/2)}

\textbf{1. Conservation de la Masse (Densité)}
\begin{equation*}
\underbrace{\frac{\partial \rho}{\partial t}}_{\text{Variation de la densité dans le temps}} + \underbrace{\frac{\partial (\rho v)}{\partial x}}_{\text{Variation du flux dans l'espace}} = 0
\end{equation*}
\begin{itemize}
    \item C'est une loi de conservation fondamentale.
    \item \textbf{Intuition :} Le nombre de véhicules ne change pas, ils ne font que se déplacer. Si plus de véhicules entrent dans une zone qu'il n'en sort, la densité augmente.
    \item Le terme $\rho v$ est le \textbf{flux} de véhicules.
\end{itemize}

\pause
\vspace{1cm}
\textbf{2. Équation de "Momentum"}
\begin{equation*}
\frac{\partial w}{\partial t} + v \frac{\partial w}{\partial x} = \frac{V_e - v}{\tau}
\end{equation*}
\begin{itemize}
    \item Décrit comment la vitesse évolue.
    \item Le terme de gauche indique que l'information de vitesse $w$ se propage avec la vitesse du trafic $v$.
    \item Le terme de droite est une \textbf{source de relaxation} : les conducteurs ajustent leur vitesse $v$ pour tendre vers une vitesse désirée $V_e$ sur un temps caractéristique $\tau$.
\end{itemize}

\end{frame}

%-------------------------------------------------
\begin{frame}
\frametitle{Comprendre les Équations (2/2)}

\textbf{Qu'est-ce que la variable $w = v + p(\rho)$ ?}

\begin{itemize}
    \item $v$ est la vitesse physique du véhicule.
    \item $p(\rho)$ est un terme de "pression" qui modélise l'anticipation des conducteurs.
    \item \textbf{Intuition :} Les conducteurs ne réagissent pas seulement à la densité locale, mais anticipent la densité devant eux. C'est ce qui différencie le trafic humain d'un simple fluide.
    \item C'est la caractéristique principale des modèles de type ARZ, qui permet d'éviter des comportements irréalistes (comme des accélérations infinies).
\end{itemize}

\pause
\vspace{1cm}
\textbf{Qu'est-ce que la vitesse à l'équilibre $V_e(\rho)$ ?}
\begin{itemize}
    \item C'est la vitesse que les conducteurs adopteraient dans des conditions de trafic stables (fluide).
    \item Elle dépend de la densité totale : plus il y a de monde, plus $V_e$ est faible.
    \item Une forme typique est : $V_e(\rho) = V_{max} \left(1 - \frac{\rho_{total}}{\rho_{jam}}\right)$
    \item $V_{max}$ est la vitesse maximale (route vide), $\rho_{jam}$ est la densité maximale (embouteillage complet).
\end{itemize}

\end{frame}

%-------------------------------------------------
\section{La Chaîne de Résolution Numérique}
%-------------------------------------------------

\begin{frame}
\frametitle{Le Passage du Continu au Discret}

Les équations que nous avons vues sont \textbf{continues} (définies en tout point $x$ et $t$). Un ordinateur ne peut pas les gérer directement.

\textbf{Solution : La Discrétisation}

\begin{enumerate}
    \item \textbf{Discrétisation Spatiale :} On découpe la route en segments (cellules) de taille $\Delta x$.
    \pause
    \item \textbf{Discrétisation Temporelle :} On avance dans le temps par petits pas de durée $\Delta t$.
\end{enumerate}

\begin{center}
    \includegraphics[width=0.8\textwidth]{grid.png} % Placeholder for grid diagram
\end{center}

On ne calcule plus la solution exacte partout, mais une \textbf{approximation} de la moyenne de la solution dans chaque cellule à chaque pas de temps.

\end{frame}

%-------------------------------------------------
\begin{frame}
\frametitle{La Méthode des Volumes Finis}

Le cœur de la résolution numérique est la méthode des volumes finis. On transforme l'EDP en une simple mise à jour :

\begin{equation*}
U_i^{n+1} = U_i^n - \frac{\Delta t}{\Delta x} \left( F_{i+1/2} - F_{i-1/2} \right)
\end{equation*}

\begin{itemize}
    \item $U_i^n$: La valeur moyenne dans la cellule $i$ au temps $n$.
    \item $U_i^{n+1}$: La nouvelle valeur que l'on cherche à calculer.
    \item $F_{i+1/2}$: Le \textbf{flux numérique} à l'interface entre la cellule $i$ et $i+1$.
\end{itemize}

\pause
\vspace{1cm}
\textbf{Le problème est maintenant réduit à une seule question : Comment calculer le flux $F$ aux interfaces ?}

\end{frame}

%-------------------------------------------------
\begin{frame}
\frametitle{Étape 1 : Reconstruction Spatiale (WENO)}

À l'intérieur d'une cellule, nous n'avons qu'une valeur moyenne. Mais pour calculer le flux à l'interface, nous avons besoin des valeurs \textbf{exactement sur le bord} de la cellule.

\begin{center}
    \includegraphics[width=0.7\textwidth]{reconstruction.png} % Placeholder for reconstruction diagram
\end{center}

\textbf{Le problème :}
\begin{itemize}
    \item Une reconstruction simple (linéaire) crée des oscillations près des chocs (embouteillages).
    \item Une reconstruction trop simple (constante) "bave" les chocs et les rend flous (diffusion numérique).
\end{itemize}

\pause
\textbf{La solution : WENO (Weighted Essentially Non-Oscillatory)}
\begin{itemize}
    \item Méthode de reconstruction d'ordre élevé (précise).
    \item Utilise plusieurs "stencils" (groupes de cellules voisines) pour deviner la valeur au bord.
    \item Attribue des poids à chaque stencil : si un stencil contient un choc, son poids devient quasi nul.
    \item \textbf{Résultat :} Une reconstruction très précise dans les zones fluides, et stable (sans oscillations) près des chocs.
\end{itemize}

\end{frame}

%-------------------------------------------------
\begin{frame}
\frametitle{Étape 2 : Le Problème de Riemann à l'Interface}

Grâce à WENO, nous avons maintenant deux valeurs à chaque interface :
\begin{itemize}
    \item $U_L$ (reconstruite depuis la cellule de gauche)
    \item $U_R$ (reconstruite depuis la cellule de droite)
\end{itemize}

\begin{center}
    \includegraphics[width=0.6\textwidth]{riemann.png} % Placeholder for Riemann problem
\end{center}

\textbf{Le Problème de Riemann :}
C'est un mini-problème mathématique qui consiste à trouver la solution exacte qui résulte de la "collision" de ces deux états $U_L$ et $U_R$. La solution nous donne le flux $F_{i+1/2}$ que nous cherchions.

\pause
\textbf{La solution du code : Solveur de Riemann "Central-Upwind"}
\begin{itemize}
    \item Un solveur approximatif, mais robuste et efficace.
    \item Il estime la vitesse de propagation des ondes vers la gauche ($a_L$) et vers la droite ($a_R$).
    \item Il calcule un flux qui est une moyenne pondérée des flux de $U_L$ et $U_R$, stabilisée par un terme de dissipation basé sur $a_L$ et $a_R$.
\end{itemize}

\end{frame}

%-------------------------------------------------
\begin{frame}
\frametitle{Étape 3 : Intégration Temporelle (SSP-RK3)}

Maintenant que nous savons calculer la divergence des flux $\left( F_{i+1/2} - F_{i-1/2} \right)$, nous pouvons mettre à jour la solution.

L'équation est devenue une Équation Différentielle Ordinaire (EDO) en temps :
\begin{equation*}
\frac{dU_i}{dt} = L(U_i) \quad \text{où } L(U_i) = -\frac{1}{\Delta x} \left( F_{i+1/2} - F_{i-1/2} \right)
\end{equation*}

\pause
\textbf{Le problème :}
\begin{itemize}
    \item Une simple mise à jour (méthode d'Euler) n'est pas assez stable pour les schémas d'ordre élevé comme WENO.
\end{itemize}

\pause
\textbf{La solution : SSP-RK3 (Strong Stability Preserving Runge-Kutta 3ème ordre)}
\begin{itemize}
    \item C'est une méthode de Runge-Kutta qui effectue 3 sous-étapes pour calculer $U_i^{n+1}$.
    \item \textbf{SSP} signifie qu'elle est conçue pour préserver la stabilité (ne pas créer d'oscillations) que le schéma spatial (WENO) a mis tant d'efforts à obtenir.
    \item C'est le partenaire idéal pour WENO.
\end{itemize}

\begin{align*}
    U^{(1)} &= U^n + \Delta t L(U^n) \\
    U^{(2)} &= \frac{3}{4}U^n + \frac{1}{4}(U^{(1)} + \Delta t L(U^{(1)})) \\
    U^{n+1} &= \frac{1}{3}U^n + \frac{2}{3}(U^{(2)} + \Delta t L(U^{(2)}))
\end{align*}

\end{frame}

%-------------------------------------------------
\section{Synthèse et Conclusion}
%-------------------------------------------------

\begin{frame}
\frametitle{La Chaîne Numérique Complète}

\begin{enumerate}
    \item \textbf{Départ :} On a l'état du trafic $U^n$ à un temps $n$.
    \pause
    \item \textbf{Reconstruction (WENO) :} Pour chaque cellule, on calcule les valeurs $U_L$ et $U_R$ aux bords à partir des moyennes des cellules voisines.
    \pause
    \item \textbf{Flux (Solveur de Riemann) :} À chaque interface, on résout le problème de Riemann entre $U_L$ et $U_R$ pour obtenir le flux numérique $F_{i+1/2}$.
    \pause
    \item \textbf{Divergence :} Dans chaque cellule, on calcule le terme $L(U) = -\frac{1}{\Delta x} (F_{i+1/2} - F_{i-1/2})$.
    \pause
    \item \textbf{Évolution (SSP-RK3) :} On utilise ce terme $L(U)$ dans les 3 étapes du SSP-RK3 pour calculer le nouvel état $U^{n+1}$.
    \pause
    \item \textbf{Répéter :} On recommence pour le pas de temps suivant.
\end{enumerate}

\vspace{1cm}
\textbf{Le tout est orchestré sur GPU pour une performance maximale, en minimisant les transferts de données entre le CPU et le GPU.}

\end{frame}

%-------------------------------------------------
\begin{frame}
\frametitle{Conclusion}

\begin{itemize}
    \item Nous avons transformé un problème physique complexe (le trafic) en un \textbf{modèle mathématique} (système d'EDP hyperboliques).
    \pause
    \item Ce modèle continu a été discrétisé pour être résolu par un ordinateur via la \textbf{méthode des volumes finis}.
    \pause
    \item La précision et la stabilité sont garanties par une combinaison de méthodes numériques avancées :
    \begin{itemize}
        \item \textbf{WENO5} pour la reconstruction spatiale.
        \item Un \textbf{solveur de Riemann} pour les flux aux interfaces.
        \item \textbf{SSP-RK3} pour l'intégration temporelle.
    \end{itemize}
    \pause
    \item Cette "chaîne numérique" permet de simuler l'évolution du trafic de manière robuste et précise, en capturant des phénomènes complexes comme les ondes de choc.
\end{itemize}

\end{frame}

\end{document}
