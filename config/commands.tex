% Define custom commands and macros for frequently used symbols or expressions
% Requires relevant math packages (amsmath, amssymb, bm - loaded in packages.tex)
% Requires xcolor package for \highlight (loaded in packages.tex, colors defined in colors.tex)

% --- Mathematical Symbols ---
\newcommand{\R}{\mathbb{R}} % Real numbers
\newcommand{\N}{\mathbb{N}} % Natural numbers
\newcommand{\Z}{\mathbb{Z}} % Integers
\newcommand{\C}{\mathbb{C}} % Complex numbers

\newcommand{\abs}[1]{\left\lvert#1\right\rvert}      % Absolute value: |x|
\newcommand{\norm}[1]{\left\lVert#1\right\rVert}    % Norm: ||x||
\newcommand{\inner}[2]{\left\langle#1, #2\right\rangle} % Inner product: <x, y>

% --- Calculus and Differential Equations ---
\newcommand{\dd}{\mathop{}\!\mathrm{d}}              % Differential d (for integrals, etc.) e.g., \int f(x) \dd x
\newcommand{\pd}[2]{\frac{\partial #1}{\partial #2}} % Partial derivative: d/dx
\newcommand{\pdd}[2]{\frac{\partial^2 #1}{\partial #2^2}} % Second partial derivative: d^2/dx^2
\newcommand{\pdm}[3]{\frac{\partial^2 #1}{\partial #2 \partial #3}} % Mixed partial derivative
\newcommand{\od}[2]{\frac{\dd #1}{\dd #2}}          % Ordinary derivative: d/dx
\newcommand{\odd}[2]{\frac{\dd^2 #1}{\dd #2^2}}     % Second ordinary derivative: d^2/dx^2

\newcommand{\grad}{\nabla}                         % Gradient symbol
\newcommand{\laplace}{\Delta}                      % Laplacian symbol

% --- Model Specific Notation (Examples - Adapt as needed based on your paper) ---
\newcommand{\popS}{S}                               % Susceptible population density
\newcommand{\popI}[1]{I_{#1}}                       % Infected population density for strain #1
\newcommand{\popN}{N}                               % Total population size (constant)
\newcommand{\domain}{\Omega}                         % Spatial domain
\newcommand{\boundary}{\partial\Omega}               % Boundary of the domain
\newcommand{\normal}{\vec{n}}                        % Outward unit normal vector

\newcommand{\diffS}{d_S}                             % Diffusion coefficient for S
\newcommand{\diffI}[1]{d_{#1}}                      % Diffusion coefficient for I_#1
\newcommand{\betaI}[1]{\beta_{#1}}                   % Transmission rate for strain #1
\newcommand{\gammaI}[1]{\gamma_{#1}}                 % Recovery rate for strain #1

\newcommand{\Rzero}{\mathcal{R}_0}                   % Basic reproduction number (multi-strain)
\newcommand{\Rzeroi}[1]{\mathcal{R}_{0,#1}}          % Basic reproduction number for strain #1
\newcommand{\Rloc}{\mathfrak{R}}                     % Local reproduction function (multi-strain)
\newcommand{\Rloci}[1]{\mathfrak{R}_{#1}}            % Local reproduction function for strain #1
\newcommand{\Rinv}[1]{\tilde{\mathcal{R}}_{#1}}      % Invasion reproduction number for strain #1

\newcommand{\SigmaSet}[1]{\Sigma_{#1}}               % Sigma sets (Sigma_0, Sigma_1, Sigma_2)

% --- Operators ---
\DeclareMathOperator*{\argmax}{arg\,max}             % Arg max
\DeclareMathOperator*{\argmin}{arg\,min}             % Arg min
\DeclareMathOperator{\supp}{supp}                   % Support of a function
\DeclareMathOperator{\esssup}{ess\,sup}             % Essential supremum
\DeclareMathOperator{\trace}{Tr}                    % Trace of a matrix/operator
\DeclareMathOperator{\diag}{diag}                   % Diagonal matrix

% --- Text Macros ---
\newcommand{\eg}{e.g.,}                             % Example abbreviation
\newcommand{\ie}{i.e.,}                             % Id est abbreviation
\newcommand{\etal}{\textit{et al.}}                 % Et al. abbreviation

% --- Commands from benintraffic.sty ---
% Highlight equation box (requires color 'lightblue' defined in colors.tex)
\newcommand{\highlight}[1]{%
\colorbox{lightblue!15}{$\displaystyle#1$}}

% Optional: Commands for numerical scheme indices (if useful for your discretization chapter)
% \newcommand{\rhonij}[3]{\rho_{#1,#2}^{#3}} % Density with space-time indices
% \newcommand{\rhoni}[2]{\rho_{#1}^{#2}}     % Density with time index
% \newcommand{\rhoij}[2]{\rho_{#1,#2}}       % Density with space index
% \newcommand{\vnij}[3]{v_{#1,#2}^{#3}}      % Velocity with space-time indices
% \newcommand{\vni}[2]{v_{#1}^{#2}}          % Velocity with time index
% \newcommand{\vij}[2]{v_{#1,#2}}            % Velocity with space index
% \newcommand{\fnij}[3]{F_{#1,#2}^{#3}}      % Flux with space-time indices

% Add more custom commands as you identify recurring patterns in your writing